\documentclass[../main/main.tex]{subfiles}
\begin{document}

\newcommand{\V}[1]{\mathbb V(#1)}
\newcommand{\I}[1]{\mathbf I(#1)}
\newcommand{\mc}[1]{\mathcal #1}

\chapter{Exercise Sheet - week 1}

\begin{ex}
Let $X\subset\vec A^m$ and $Y\subset \vec A^n$ be algebraic sets. Prove that $X\times Y\subset\vec A^{m+n}$ is an algebraic set. \\
%
\textit{Solution:}
%
Let $a_1,\dots,a_m$ be the coordinates over $\vec A^m$ and $b_1,\dots,b_n$ be the coordinates over $\vec A^n$. There are two subsets $S_1\subset k[a_1,\dots,a_m]$ and $S_2\subset k[b_1,\dots,b_n]$ such then $\V{S_1}=X$ and $\V{S_2}=Y$. Then if we define $S=\{f\cdot g\vert f\in S_1, g\in S_2\}\subset k[a_1,\dots,a_m,b_1,\dots,b_n]$ we have $X\times Y=\V S\subset\vec A^{m+n}$.

\end{ex}

\begin{ex}
Let $X,Y\subset\vec A^n$ be algebraic sets. Prove the following equalities:
\begin{enumerate}[label=(\alph*)]
\item $\I{X\cup Y}=\I X\cap \I Y$
\item $\I{X\cap Y}=\sqrt{\I X+\I Y}$
\end{enumerate}
where for two ideals $\mc I, \mc J$ we denote by $\mc I+\mc J$ the ideal generated by the union $\mc I\cup\mc J$.\\
Find an example where $\I{X\cap Y}$ and $\I X\cap \I Y$ are different. Can you give a geometric explenation of why we have an inequality in this case?\\
%
\textit{Solution:}
%
(a),``$\subset$'' If $f\in \I{X\cup Y}$ then $f(p)=0$ for each $p\in X\cup Y$. In particular $f(p)=0$ for each $p\in X$, i.e. $f\in\I X$; and $f(p)=0$ for each $p\in Y$, i.e. $f\in\I Y$; so $f\in  \I X\cap \I Y$.\\
``$\supset$'' If $f\in  \I X\cap \I Y$ then $f(p)=0$ for each $p\in X$ and $f(p)=0$ for each $p\in Y$. Then $f(p)=0$ for each $p\in X\cup Y$, i.e. $f\in \I{X\cup Y}$.\\
(b),``$\supset$'' if $f\in \sqrt{\I X+\I Y}$, then exists $N\geq1$ such that $f^N=g+h$ where $g\in\I{X}$ and $h\in\I{Y}$, and this implies that $f^N\in \I{X\cap Y}$. For any point $p$ condition $f^N(p)=0$ implies $f(p)=0$, hence $f\in \I{X\cap Y}$. \\
``$\subset$'' First, recall that $\sqrt{\I X+\I Y}=\I{\V{{\I X+\I Y}}}$. If $p\in \V{{\I X+\I Y}}$, then for each $f\in \I X$ and $g\in\I{Y}$ we have $f(p)=0=g(p)$, and this means $p\in X\cap Y$. We conclude using inclusion reversing propriety: $\V{{\I X+\I Y}}\subset X\cap Y$ implies $\I{X\cap Y}\subset\I{\V{{\I X+\I Y}}}=\sqrt{\I X+\I Y}$.\\
The ideal $\I{X\cap Y}$ describes the intersection of algebraic sets $X$ and $Y$ while $\I X\cap \I Y$ describes them union. For example, if we consider two different sets $X=\{a\}\subset\vec A^1$ and $Y=\{b\}\subset\vec A^1$ and following polynomials: $f=x-a$, $g=x-b$ and $h=(x-a)(x-b)$. We have $f,h\in \I X$ and $g,h\in\I Y$, while only $h\in \I X\cap \I Y$. Since $X\cap Y=\emptyset$, we have $f,g,h\in\I{X\cap Y}$.
\end{ex}

\begin{ex}
Compute the ideals of the following algebraic sets in $\vec A^2(\CC)$:
\begin{enumerate}[label=(\alph*)]
\item $X_1=\{(1,0),(0,1)\}$
\item $X_2=\{(1,0),(0,1),(0,0)\}$
\item $X_3=\{(1,0),(0,1),(\frac12,\frac12)\}$
\end{enumerate}
What is the minimal number of polynomials you need to generate $\I{X_1}$, $\I{X_2}$ and $\I{X_3}$ respectively?\\
%
\textit{Solution:}
%
Let's define following sets of plynomals:
\begin{enumerate}[label=(\alph*)]
\item $S_1=\{(x-1),(x-i)\}$
\item $S_2=\{(x-1),(x-i),(x)\}$
\item $S_3=\{(x-1),(x-i),(x-\frac12-\frac12i)\}$
\end{enumerate}
Then $\I{X_1}=(S_1)$, $\I{X_2}=(S_2)$ and $\I{X_3}=(S_3)$. It's clear that minimal number of polynomials required to generate ideals is 2 for $X_1$ and 3 for $X_2$ and $X_3$.
\end{ex}

\begin{ex}
Let $X$ be the union of the coordinate axis in $\vec A^n$. Find generators for the ideal of $X$, how many polynomials do we need?\\
%
\textit{Solution:}
%
The union of the coordinate axes in $\vec A^n$ is the locus where at least one coordinate is zero, therefore $\I{X}=(\{(x_1),(x_2),\dots,( x_n)\})$, where $x_i$ is the coordinate of the $i$-th axis; hence we need only one generator for $\I{X}$.

\end{ex}

\begin{ex}
Let $X=\{(t,t^2,t^3):t\in k\}\subset \vec A^3$. Prove that $X$ is an irreducible algebraic set and find generators for its ideal $\I X$. Show that the dimension of $X$ is one, i.e. $X$ is an irreducible curve.\\
%
\textit{Solution:}
%
The algebraic set is the locus where $y=x^2$ and $z=x^3$, with $\{x,y,z\}$ coordinate system for $\vec A^3$. Then $\I X=(\{(y-x^2),(z-x^3)\})$. This ideal is principal and hence is prime, therefore $X$ is irreducible.  
In order to find algebraic subsets of $X$, lets try to insert another polynomial in the generator of $\I X$, namely $f\in k[x,y,z]$. Notice first of all that coordinates $y$ and $z$ are already fixed by $x$, therefore in order to obtain some subset $X_0=\V{(\{(y-x^2),(z-x^3), f(x,y,z)\})}\neq\emptyset$, we require $f\in k[x]$. Since such function $f$ will have a finite number of zeros, $X_0$ will be a collection of points, with zero dimension (since a point can't have any subset). We conclude that the maximum descending chain of subset is $X\subsetneqq X_0\subsetneqq\emptyset$, where $X_0$ is a collection of points, hence $X$ is an irreducible curve.
\end{ex}

\begin{ex}
Let $k=\CC$. Decompose into irreducible components the following algebraic sets $X,Y\subset\vec A^3$ and determine the prime ideals of their irreducible components:
\begin{enumerate}[label=(\alph*)]
\item $X$ defined by $x_1^2+x_2^2+x_3^2=x_1^2-x_2^2-x_3^2+1=0$
\item $Y$ defined by $x_1^2-x_2x_3=x_1x_3-x_1=0$
\end{enumerate}
%
\textit{Solution:}
%
(a) Let $\I X=(\{(x_1^2+x_2^2+x_3^2),(x_1^2-x_2^2-x_3^2+1)\})$. We can define subsets $X_1,X_2\subset X$ through ideals $\I{X_1}=(\{(x_2^2+x_3^2-\frac i2),(x_1-\frac i{\sqrt2})\})$ and $\I{X_2}=(\{(x_2^2+x_3^2+\frac i2),(x_1-\frac i{\sqrt2})\})$. Notice that neither $(x_2^2+x_3^2+\frac i2)$ nor $(x_1\pm\frac i{\sqrt2})$ can be decomposed further in other polynomials with smaller degree. If we take $f\in k[x_1,x_2,x_3]/\I{X_1}$ and $g\in k[x_1,x_2,x_3]$ with $f g\in \I{X_1}$, then for some $a,b\in k[x_1,x_2,x_3]$ we have $fg=a(x_2^2+x_3^2-\frac i2)+b(x_1-\frac i{\sqrt2})$ and $g=(a/f)(x_2^2+x_3^2-\frac i2)+(b/f)(x_1-\frac i{\sqrt2})$ for some $(a/f),(b/f)\in k[x_1,x_2,x_3]$, and then $g\in  \I{X_1}$. This means that  $\I{X_1}$ is prime, and same proof holds also for $ \I{X_2}$. Therefore $X_1$ and $X_2$ are affine varieties. Since $\I X=\I{X_1}\cap\I{X_2}$, we conclude $X=X_1\cup X_2$, where $X_1$ and $X_2$ are irreducible components of $X$.\\
(b) Let $\I Y=(\{(x_1^2-x_2x_3),(x_1x_3-x_1)\})$. We can define subsets $Y_1,Y_2,Y_3\subset Y$ through ideals $\I{Y_1}=(\{(x_1),(x_2)\})$, $\I{Y_2}=(\{(x_1),(x_3)\})$ and $\I{Y_3}=(\{(x_1^2-x_2),(x_3-1)\})$. Analogously to (a), we can prove that $Y=Y_1\cup Y_2\cup Y_3$, where $Y_1$, $Y_2$ and $Y_3$ are irreducible components of $Y$.\\

\end{ex}

\begin{ex} Let $Y$ be a subset of a topological space $X$. Show that $Y$ is irreducible if and only if the closure $\overline Y$ of $Y$ in $X$ is irreducible. \\
%
\textit{Solution:}
%
``$\Rightarrow$'' Let's take $Y\subset X$ irreducible subset of the topological space $X$. Let $\overline Y\subset X$ be the smallest closed subset that contains $Y$. Suppose that $\overline Y=\overline Y_1\cup \overline Y_2$, then $Y=(Y\cap \overline Y_1)\cup(Y\cap \overline Y_2)$, and since $Y$ is irreducible then it must be contained either in $\overline Y_1$ or $\overline Y_2$. Assume that $Y\subset \overline Y_1$. Since $\overline Y_1\subset \overline Y$ is a closed subset that contains $Y$ and $\overline Y$ is the smallest closed subset that contains $Y$ then $\overline Y=\overline Y_1$ and than $\overline Y$ is not reducible. \\
%
``$\Leftarrow$'' Let's take $Y\subset X$ reducible subset of the topological space $X$, and let $Y=Y_1\cup  Y_2$ with $Y_i\subsetneqq Y$, $i=1,2$,  subsets of $Y$  closed in the subspace topology. Consider $\overline Y$, $\overline Y_1$ and $\overline Y_2$ to be the closure respectively of $Y$, $Y_1$ and $Y_2$; then $\overline Y_1\cup\overline Y_2\supset \overline Y$ because $\overline Y$ is the smallest closed set containing $Y$, and on the other side $\overline Y_1\cup\overline Y_2\subset \overline Y$ otherwise $\overline Y\cap\overline Y_i$ would be a closure of $Y_i$ smaller that $\overline Y_i$ either for $i=1$ or $i=2$. Finally, notice that $Y_i=\overline Y_i\cap Y$ and  $Y=\overline Y\cap Y$, so $Y_i\subsetneqq Y$ implies $\overline Y_i\subsetneqq \overline Y$. Then we can conclude that  $\overline Y=\overline Y_1\cup\overline Y_2$ is reducible. 

\end{ex}

\begin{ex}
Let $f:X\to Y$ be a continuous map and let $W\subset X$ be an irreducible subset of $X$. Prove that the image of $W$ is irreducible. \\
%
\textit{Solution:}
%
Using the result of the previous exercise, we just have to prove the statement in the case of $W$ and $V=f(W)$ closed subsets. Then, suppose $V=V_1\cup V_2$ and $W_i=(f^{-1}(V_i))\cap W$ for $i=1,2$. Then $W_1$ and $W_2$ are closed subsets of $W$ and $W=W_1\cup W_2$, hence we can assume $W=W_1$. Then $V=f(W)=f(W_1)=V_1$, so $V$ is irreducible.

\end{ex}





























\end{document}