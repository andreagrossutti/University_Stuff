\documentclass[../main/main.tex]{subfiles}
\begin{document}

\chapter{Preliminaries}
\section{Quick review of Special Relativity}

Here we expose a quick review of Special Relativity in order to set the notations.

\skipline
Fundamental principles of Special Relativity are followings:
\begin{enumerate}
\item All inertial reference frames are physically equivalent. There is no way to distinguish between different inertial frames in the sense that there is no preferred one. 
\item There exists universal (dimensional) constant: $c\simeq3\times10^8\text{m/s}$, i.e. the speed of massless particles.
\end{enumerate}

\skipline
In order to implement these features basic ingredients are
\begin{enumerate}
\item Space and Time form a unique concept called \textbf{spacetime}. 
\item A spacetime is a collection of points called \textbf{event}.
\item Each inertial frame is associated with a set of \emph{space time coordinates}. Each events is specified through coordinate system of a fixed initial frame. \[x^\mu=(x^0,x^1,x^2,x^3)\equiv(x^0,x^i)\equiv(ct,x,y,z)\equiv(ct,\vec x)\]
Usually $x,y,z$ are assumed to be Cartesian coordinates.


Given 2 events $A$ and $B$ in  spacetime 




\tikzset{every picture/.style={line width=0.75pt}} %set default line width to 0.75pt     
\[
\begin{tikzpicture}[x=0.75pt,y=0.75pt,yscale=-1,xscale=1]
%uncomment if require: \path (0,300); %set diagram left start at 0, and has height of 300

%Straight Lines [id:da9013720818372855] 
\draw    (195,224) -- (195,141) ;
\draw [shift={(195,139)}, rotate = 450] [color={rgb, 255:red, 0; green, 0; blue, 0 }  ][line width=0.75]    (10.93,-3.29) .. controls (6.95,-1.4) and (3.31,-0.3) .. (0,0) .. controls (3.31,0.3) and (6.95,1.4) .. (10.93,3.29)   ;
%Straight Lines [id:da7261382621997787] 
\draw    (195,224) -- (302.5,224) ;
\draw [shift={(304.5,224)}, rotate = 180] [color={rgb, 255:red, 0; green, 0; blue, 0 }  ][line width=0.75]    (10.93,-3.29) .. controls (6.95,-1.4) and (3.31,-0.3) .. (0,0) .. controls (3.31,0.3) and (6.95,1.4) .. (10.93,3.29)   ;
%Straight Lines [id:da22577623387415535] 
\draw    (195,224) -- (141.99,271.66) ;
\draw [shift={(140.5,273)}, rotate = 318.03999999999996] [color={rgb, 255:red, 0; green, 0; blue, 0 }  ][line width=0.75]    (10.93,-3.29) .. controls (6.95,-1.4) and (3.31,-0.3) .. (0,0) .. controls (3.31,0.3) and (6.95,1.4) .. (10.93,3.29)   ;
%Straight Lines [id:da04380450147553239] 
\draw [color={rgb, 255:red, 65; green, 117; blue, 5 }  ,draw opacity=1 ][line width=0.75]    (219.5,203) -- (276.5,169) ;
\draw [shift={(276.5,169)}, rotate = 329.18] [color={rgb, 255:red, 65; green, 117; blue, 5 }  ,draw opacity=1 ][fill={rgb, 255:red, 65; green, 117; blue, 5 }  ,fill opacity=1 ][line width=0.75]      (0, 0) circle [x radius= 1.9, y radius= 1.9]   ;
\draw [shift={(219.5,203)}, rotate = 329.18] [color={rgb, 255:red, 65; green, 117; blue, 5 }  ,draw opacity=1 ][fill={rgb, 255:red, 65; green, 117; blue, 5 }  ,fill opacity=1 ][line width=0.75]      (0, 0) circle [x radius= 1.5, y radius= 1.5]   ;

% Text Node
\draw (120,137) node    {$$};
% Text Node
\draw (283,258) node    {$x,y,z$};
% Text Node
\draw (177,141) node    {$ct$};
% Text Node
\draw (210,207) node    {$A$};
% Text Node
\draw (280,157) node    {$B$};


\end{tikzpicture}\]

their distance is $\Delta x^\mu=x^\mu_B-x_A^\mu$. We introduce the (squared) \emph{Minkowski distance}
\[\Delta s^2=\eta_{\mu\nu}\Delta x^\mu\Delta^\nu\quad\text{where}\quad\eta_{\mu\nu}=\begin{pmatrix}-1&&&\\&1&&\\&&1&\\&&&1\end{pmatrix}\]
where $\eta_{\mu\nu}$ is \emph{Minkowski metric}.
This induces the \emph{line element}
\[\de s^2=\eta_{\mu\nu}\de x^\mu\de x^\nu\]
This element is scalar quantity and therefore does not depends on the specific inertial frame.

The quantity $\Delta s^2$ has an intrinsic meaning
\[\begin{cases}\begin{alignedat}{2}
&\Delta s^2>0\quad&&:\quad \Delta x^\mu\text{ is \emph{space-like} vector}\\
&\Delta s^2=0\quad&&:\quad \Delta x^\mu\text{ is \emph{time-like} vector}\\
&\Delta s^2<0\quad&&:\quad \Delta x^\mu\text{ is \emph{light-like/null} vector}
\end{alignedat}\end{cases}\]
Space-like vector means that exists different frames were two events are simultaneous. Time-like vector means that exists different frames where two events have same space coordinates but they happen at different times. Light-like vectors means that two events may be connected by a light signal.

\item Allowed transformations for spacetime vectors must preserve the line element: $\Delta\tilde s^2=\Delta s^2$. These transformations are the \emph{Poincaré Transformations}

\[x^\mu\quad\to\quad\tilde x^\mu={\Lambda^\mu}_\nu x^\nu+a^\mu\qquad\text{with}\qquad{\Lambda^\rho}_\mu{\Lambda^\sigma}_\nu\eta_{\rho\sigma}=\eta_{\mu\nu}\]


\end{enumerate}

\skipline
Once we have reformulated notions of space and time, we have to reformulate law of physics in such a way they does not depends on the reference frame.

Trajectories of point like-particles are associated to curved \emph{wordlines} in space time and described evolution of events. Mathematically they are described by maps from $\RR$ into a set of four functions: $\lambda\in\RR\to x^\mu(\lambda)$. Near if we consider nearby events separated by infinitesimal shift we can obtain infinitesimal variation of coordinates:
\[\de x^\mu(\lambda)\equiv x^\mu(\lambda+\de\lambda)-x^\mu(\lambda)=\frac{\de x^\mu(\lambda)}{\de\lambda}\de\lambda\]
Since no particles can move at a speed higher then light this implies that $\de s^2$ must be time-like. Notice that choice of parameter $\lambda$ is free. One possible choice of this parameter is the \emph{(differential) proper time}:
\begin{align*}
\de\tau\equiv\sqrt{-\de s^2}
&=\de\lambda\sqrt{-\eta_{\mu\nu}\dot x^\mu(\lambda)\dot x^\nu(\lambda)}
=c\,\de t\sqrt{1-\frac{v^2}{c^2}}\equiv\frac{c\,\de t}\gamma
\end{align*}
where third step holds if $\lambda\equiv t$. If we define $\beta\equiv v/c$ \footnote{This is the speed in natural units, i.e. in units of $\beta$. If we set $c=1$ then $v=\beta$.}, then $\gamma=1/\sqrt{1-\beta^2}$ is called \emph{Lorentz factor}. Notice that last step implies time dilatation at higher velocities. For $\lambda=t$ we obtain
\[\tau=c\int\de t\sqrt{1-\frac{v^2}{c^2}}\]
i.e. with this definition the proper time has dimension of a length $[\tau]=L$. Physically the proper times it's the time measured by a clock moving along the trajectory. 

Proper time allow us to define a vector called \emph{4-velocity} that can be identified as relativistig generalization of velocity. Namely:
\[u^\mu(\tau)=\frac{\de x^\mu(\tau)}{\de\tau}=\left(\gamma,\gamma\frac{\vec v}{c}\right)\]
Notice 
\[u^\mu u_\mu=-1\]
i.e. is a time-like vector. Moreover, this vector has only three degrees of freedom, since one component is fixed by previous propriety.


Now we can define the generalization of acceleration, \emph{4-acceleration}, as follows
\[\alpha^\mu(\tau)=\frac{\de u^\mu(\tau)}{\de \tau}\]
Notice that, as we expected, 4-acceleration is orthogonal to 4-velocity
\[u_\mu\alpha^\mu=0\]
and this implies that $\alpha^\mu$ is a space-like since it is orthogonal to a time-like vector.

This proves a relativistic generalization of distance, speed and acceleration. Also laws of dynamic can be generalizated, in particular if we define the \emph{four-force} $f^\mu$ as the generalization of force we can obtain the
 \emph{Relativistic Second Newton's law}:
\[mc\alpha^\mu\equiv\frac{\de p^\mu}{\de\tau}=f^\mu\]
where we used four acceleration or equivalently the generalization of newtonian momentum, \emph{4-momentum},
\[p^\mu\equiv mcu^\mu=\p{\frac Ec,\vec p}\]

For example, for Lorentz force
\[\vec F_L=e\p{\vec E+\frac1c\vec v\cross\vec B}\]
can be generalizzated in a manifestly covariant way into\footnote{Here is evident that this formula does not change under Poincaré transformations.}
\[f^\mu_L=\frac ecF^{\mu\nu}u_\nu\hspace{1.5cm}\text{with}\quad F^{\mu\nu}=
\begin{pmatrix}0&E_1&E_2&E_3\\-E_1&0&B_3&-B_2\\-E_2&-B_3&0&B_1\\-E_3&B_2&-B_1&0\end{pmatrix}\]
where $F^{\mu\nu}$ is the \emph{EM-Tensor}.

We can also rewrite Maxwell equations into two covariant equation
\[\partial_\mu F^{\mu\nu}=-\frac{4\pi}{c}j^\nu\quad,\quad\partial_{[\mu}F_{\nu\rho]}=0\]
where the former, inhomogeneous, shows the \emph{4-current} $j^\mu=(c\rho, \vec j)$. The second equation, homogeneous, exhibits total antisymmetrized indexes\footnote{$\partial_{[\mu}F_{\nu\rho]}=\partial_{\mu}F_{\nu\rho}+\partial_\rho F_{\mu\nu}+\partial_\nu F_{\rho\mu}$.}. Each of these equations contains 2 independent equations.


We can conclude saying that all possible interactions can be written in a covariant way, except from gravitation. In order to include this force General Relativity has been developed.











\end{document}