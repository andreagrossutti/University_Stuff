\documentclass[../main/main.tex]{subfiles}
\begin{document}

\chapter{The need of a non-perturbative approach}

We saw in the previous chapter that all the interesting physical quantities can be computed, both in high-energy physics and in condensed matter physics, using the perturbative approach. In this chapter we'll see why what we said doesn't really work so good, and in some instances we need a non-perturbative approach. We already anticipated some problems in the introduction, now we'll go more in detail. 

\section{The asymptotic series}

\subsubsection{Divergent quantities}

Let's start from the simple example provided by a massive theory with quartic coupling in $d=0$. Consider the following quantity, which naturally emerges in perturbative computations:
\begin{eq}\label{eq:pert-probl-int1}
	\frac{\displaystyle \int\de x\,e^{-\frac\alpha2x^2}e^{-\lambda x^4}}{\displaystyle\int\de x\,e^{-\frac{\alpha}2x^2}}
	\tfor \lambda>0 \tcomma \Re\alpha>0
\end{eq}
Such quantity is clearly smaller than 1, since the numerator contains the integral of a function which is everywhere smaller than the function in the integral present in the denominator. Nevertheless if one applies the perturbative prescription, the previous quantity is substitute by 
\begin{eq}\label{eq:pert-probl-int2}
	\frac{\displaystyle\sum_{n=0}^\infty\frac1{n!}\int\de x\,e^{-\frac\alpha2x^2}(-\lambda x^4)^n}{\displaystyle\int\de x\,e^{-\frac{\alpha}2x^2}}
\end{eq}
But if one tries to compute explicitly the coefficient of the series
\begin{eq}
	\frac{\displaystyle\int\de x\,e^{-\frac\alpha2x^2}(\lambda x^4)^n}{\displaystyle\int\de x\,e^{-\frac{\alpha}2x^2}}
	=\lambda^n\frac{(4n)!}{(2n)!}\frac1{2^{2n+1}}\frac1{\alpha^{2n}}
	\geq \frac{\lambda^n}2\frac{((2n)!)^2}{(2n)!}\frac1{(2\alpha)^{2n}}
	=\frac{\lambda^n}2 \frac{(2n)!} {(2\alpha)^{2n}}
	\geq \frac12\left(\frac\lambda{(2\alpha)^2}\right)^n(n!)^2
\end{eq}
one gets that the series eq.~\eqref{eq:pert-probl-int2} is absolutely divergent
\begin{eq}
	\sum_{n=0}^\infty\left\vert\frac{(-\lambda)^n}{n!}\frac{\displaystyle\int\de x\,e^{-\frac\alpha2x^2} x^{4n}}{\displaystyle\int\de x\,e^{-\frac{\alpha}2x^2}}\right\vert
	\geq \sum_{n=0}^\infty\half\left(\frac\lambda{(2\alpha)^2}\right)^n n!=+\infty
\end{eq}

Obviously we saw that the perturbative prescription in this case didn't worked, but can be proved that the same issue appear also in higher dimension. Such problem is actually more general, indeed all the perturbation series needed to compute the collerator using Gell-Mann Low formula are divergent. 

Nevertheless is well known that usually the perturbative approach works (for instance in QED it works very well, and allows to obtain extremely precise predictions) hence we wonder if it is somehow possible to obtain the right expression of the original correlator using the coefficients of the perturbative series. In other words we want to know if, although divergent, could the perturbative series at least determine uniquely the corresponding correlation function. 

\subsubsection{The asymptotic series}

One says that a series $\sum_{n=0}^\infty a_n\lambda^n$, $\lambda>0$, is \emph{asymptotic to a function} $f(\lambda)$ if
\begin{eq}
	\lim_{\lambda\to0^+}\frac{\displaystyle\bigg|f(\lambda)-\sum_{n=0}^Na_n\lambda^n\bigg|}{\lambda^N}=0
	\tforall
	N>0
\end{eq}
This means that the absolute difference between $f(\lambda)$ and the truncated series at order $N$ is $O(\lambda^{N+1})$ so that for $\lambda\ll1$ small enough we can make the difference as mild as we prefer. 

Apparently with the increase of the order of perturbation the truncated series approximate better and better $f(\lambda)$. This naive idea is obviously wrong, due to the fact that increasing the perturbation order one should also decrease the value of $\lambda$. Indeed if one fixes the value of $\lambda$ and increases the perturbation order $N$, the absolute difference between $f(\lambda)$ and the truncated series initially decreases, but eventually it reaches its minimum and starts to increase again, diverging as $N\to\infty$. At higher perturbative orders one should take values of $\lambda$ smaller and smaller to make the correction negligible. The fact that we can take $\lambda\ll1$ to make the difference negligible doesn't solve our problem, since in physical applications the role of $\lambda$ is done by $\hbar$, and decreasing its value below its physical value is meaningless. 

\subsubsection{Non-analytical contributions}

The second problem is that any series asymptotic to a function is also asymptotic to infinitely many distinguished functions. For instance, if a series is asymptotic to some $f(\lambda)$ it is also asymptotic to $f(\lambda)+e^{-\frac1{\lambda\alpha}}$, since
\begin{eq}
	\lim_{\lambda\to0^+}\frac{\displaystyle e^{-\frac1{\lambda\alpha}}}{\lambda^N}=0
	\tforany
	N\in\N
	\tand
	a>0
\end{eq}
This is due to the fact that the series expansion of $e^{-\frac1{\lambda\alpha}}$ for $\lambda\ll1$ has all the coefficients equal to zero. 

Unfortunately terms in the form $e^{-\frac1{\lambda\alpha}}$ are exactly the contributions that arises if there are non-trivial topological configurations of the fields in the correlation function.

This makes completely impossible any unique reconstruction of the original function starting from its asymptotic series, since in any case the reconstructed function would be defined up to non-analytical terms which didn't appear in its asymptotic series. 

\subsubsection{Is the perturbative series asymptotic to the correlation function?}

Even if it's impossible to reconstruct the correlator from its asymptotic series, at lest it has been shown that in many QFT where the ultraviolet renormalization does not involve coupling constants (the so called \emph{super-renormalizable theories}) the renormalized perturbation series is asymptotic to the non-perturbative defined correlation function (e.g. QED for $d<3+1$ and $\phi^4$ in $d<3+1$).

Nevertheless if our theory is \emph{renormalizable} (hence not super-renormalizable) in the only case we have almost completely rigorous control of the RQFT, i.e. $\phi^4$ in $d=3+1$, it has been proved that the renormalized perturbation series is not asymptotic to the non-perturbative correlator. Even if we don't have a rigorous proof, this seems the case also for QED in $d=3+1$. 

\subsubsection{Borel resummation}

In some cases, called \emph{Borel resummable theories}, even if the perturbative series is divergent it is possible to resum it via \emph{Borel resummation} obtaining a finite result. 

The idea is ot introduce the identity
\begin{eq}
	1=\frac1{n!}\int_0^\infty x^ne^{-x}\de x
\end{eq}
inside the perturbative series
\begin{eq}
	\sum_{n=0}^\infty a_n\lambda^n=\int_0^\infty\sum_{n=0}^\infty\frac{a_n}{n!}(x\lambda)^ne^{-x}\de x
\end{eq}
In some instances, the introduction of the factor $\frac1{n!}$ inside the perturbative coefficients makes the new series converge: $\sum_n\frac1{n!}a_n(x\lambda)^n$ might converge even if $\sum_na_n\lambda^n$ does not. If the new coefficients are smooth enough to make the integral, we are then able to obtain the resummed series. 

The first problem is that the resummed series is still not sensible to non-analytical terms (they cannot be reconstructed just resumming the analytical contributions), moreover a-posteriori one has to check that the resummed series is asymptotic to the non-perturbative result (up to non-analytical terms), since the right convergence is not ensured in general. 

\subsubsection{Resurgence}

We just mention that there is a very recent technique called \emph{resurgence} which applies to Quantum Mechanics and allows to obtain non-perturbative results just using perturbative techniques. Is still unknown whether such technique can be implemented or not also in QFT.

%%%%%%%%%%%%%%%%%%%%%%%
%%%%%%%% LECTURE 5 %%%%%%%%
%%%%%%%%%%%%%%%%%%%%%%%

\section{The Källen-Lehmann representation and the Lehmann representation}

The second problem of the perturbative approach is related to the limit $\epsilon\to0$ of the IR cutoff of the relation
\begin{eq}	
	\ueid(t)\ophi_\tin(\vec x,t)\uei(t)=\ophi(\vec x, t)
\end{eq}
Indeed the limit $\epsilon\to0$ should be taken in a way compatible with physical properties of the interacting theory, in particular translational invariance and, in high energy physics, the Poincaré invariance. 

In order to exploit the problem we first need to introduce the spectral representation of the Green functions (the same we introduced for Matsubara and reatarded correlators, but from a more general and deep point of view). 

\subsubsection{Källen-Lehmann representation (relativistic case)}

\textsf{\cite[Section 9.3]{Greiner_1996}}\\

If $\hs$ is the Hilbert space of the interacting theory, in order to have translational and Poincaré invariance, some properties has to be satisfied: in particular we need in $\hs$
\begin{enumerate}[label=(\arabic*)]
	\item A unitary representation of the covering of the restricted Poincaré group $\cpoinc$. Let's denote by $\op P^\mu$ the corresponding generators of space-time translations.
	\item A vacuum vector $\ket0$ invariant under the representation $U(a)$, $a\in\R^{d+1}$, of spacetime translations ($d=3$) (due to homogeneity of spacetime).
	\item The spectrum of $\op P^\mu$ is contained in the forward light cone, $\spec(\op P^\mu)\subseteq V_0^+$.
	\item The operator $\ophi$ should transform with an irreducible representation of $\cpoinc$ under $U(a)$. In particular assuming that $\ophi$ is scalar we have
	\begin{eq}
		U(a)\ophi(x)U(a)^\dagger=\ophi(x-a)
	\end{eq}
\end{enumerate}
From $(1)$ we get that exists a Dirac completeness $\ket\alpha$ of (generalized) eigenvectors of $\op P^\mu$ with eigenvalues $p_\alpha$ (for simplicity we write $\sum_\alpha\ket\alpha\bra\alpha=\id$ also if $\op P^\mu$ has continuum spectrum).

Consider the 2-points function of a scalar RQFT
\begin{eq}\label{eq:2-point-func-spec-repr}
	\bra0\ophi(x)\ophi(y)\ket0
	&=\sum_\alpha\bra0\ophi(x)\ket\alpha\bra\alpha\ophi(y)\ket\alpha
	\overset{(4)}= \sum_\alpha \bra0U(x)\ophi(0)U^\dagger(x)\ket\alpha\bra\alpha U(y)\ophi(0)U^\dagger(y)\ket0\\
	&=\sum_\alpha e^{-ip_\alpha(x-y)}|\bra\alpha\ophi(0)\ket0|^2
	\overset{d=3}= \int\frac{\de^4q}{(2\pi)^3}\,\rho_+(q)e^{-iq(x-y)}
\end{eq}
where in the last step we used the identity $1=\int\de q\,\delta(q-p_\alpha)$ and we defined the Fourier transform (up to a factor $2\pi$) of the 2-point function
\begin{eq}
	\rho_+(q)=(2\pi)^3\sum_\alpha \delta(q-p_\alpha) |\bra0\ophi(0)\ket\alpha|^2
\end{eq}
which has some interesting properties:
\begin{enumerate}[label=(\alph*)]
	\item $\rho_+(q)\geq0$;
	\item $\rho_+(q)=0$ if $q\not\in\overline V_0^+$, thanks to $(3)$, the bar over the forward light cone indicate its closure;
	\item $\rho_+(\Lambda q)=\rho_+(q)$, for $\Lambda\in\loren$, thanks to $(4)$.
\end{enumerate}
Therefore from all these conditions we get that the most general form for $\rho_+$ is
\begin{eq}\label{eq:rho_+_sigma}
	\rho_+(q)=\sigma(q^2)\theta(q^0)+c\,\delta(q)
	\twith \sigma(q^2)=0 \tif q^2<0 \tand c \text{\ constant}
\end{eq}
The delta function is introduced to compensate the ambiguity due to the sign of $q^0$ in $\theta(q^0)$ when $q^0=0$. Notice that property (a), $\rho_+(q)\geq0$, implies that $\sigma(q^2)$ is a semi-definite positive function, i.e. $\sigma(q^2)\geq0$ for any value of $q\in\overline V_0^+$. 

Let's introduce the \emph{spectral function}\footnote{Notice that sometimes $\rho(q)$ is defined with an additional factor $2\pi$ in such a way that it is exactly the Fourier transform of $\bra0[\ophi(x),\ophi(y)]\ket0$. In some other cases it is defined (equivalently to eq.~\eqref{eq:spectral_func}) as 2 times the immaginary part of the retarded correlator, using the relations
\begin{eq}\label{eq:spectral_func}
	\fourier(-i\theta(t))(\omega)=\frac1{\omega+i\delta}
	\tand
	\Im\frac1{\omega+i\delta}=\pi\delta(\omega)
\end{eq}
where $\omega=q_0$.}
\begin{eq}
	\rho(q)=\rho_+(q)-\rho_+(-q)
\end{eq}
which due to eq.~\eqref{eq:2-point-func-spec-repr} is the Fourier transform (up to a factor $2\pi$) of $\bra0[\ophi(x),\ophi(y)]\ket0$: using eq.~\eqref{eq:2-point-func-spec-repr} we get
\begin{eq}\label{eq:2-point-func-spec-repr-comm}
	\bra0[\ophi(x),\ophi(y)]\ket0
	&=\bra0\ophi(x)\ophi(y)\ket0-\bra0\ophi(y)\ophi(x)\ket0\\
	&=\int\frac{\de^4q}{(2\pi)^3}\,\rho_+(q)e^{-iq(x-y)}-\int\frac{\de^4q}{(2\pi)^3}\,\rho_+(q)e^{-iq(y-x)}\\
	&=\int\frac{\de^4q}{(2\pi)^3}\,\rho_+(q)e^{-iq(x-y)}-\int\frac{\de^4q}{(2\pi)^3}\,\rho_+(-q)e^{-iq(x-y)}\\
	&=\int\frac{\de^4q}{(2\pi)^3}\,\rho(q)e^{-iq(x-y)}
\end{eq}
Using expression eq.~\eqref{eq:rho_+_sigma} we get
\begin{eq}\label{eq:rho_sigma}
	\rho(q)=\rho_+(q)-\rho_+(-q)=\sigma(q^2)\theta(q^0)+\cancel{c\,\delta(q)}-\sigma(q^2)\theta(-q^0)-\cancel{c\,\delta(-q)}=\sign(q^0)\sigma(q^2)
\end{eq}

If $\ophi$ obeys the CCR $[\ophi(\vec x,t),\dot\ophi(\vec y,t)]=i\delta(\vec x-\vec y)$ then $\rho$ satisfies the \emph{sum sule}
\begin{eq}\label{eq:sum_rule_spect_func}
	\int_{-\infty}^{+\infty}\de q^0 \,q^0\rho(q)=1
\end{eq}
Indeed consider the identity
\begin{eq}
	\int\frac{\de^3q}{(2\pi)^3}\,e^{i\vec q\cdot(\vec x-\vec y)}
	&=\delta(\vec x-\vec y)
	=-i\bra0[\ophi(\vec x,t),\dot\ophi(\vec y,t)]\ket0
	=-i\partial_{y^0}\bra0[\ophi(x),\ophi(y)]\ket0 \big|_{x_0=t \atop y_0=t}=\\
	&=-i\partial_{y^0} \int\frac{\de^4q}{(2\pi)^3}\,\rho(q)e^{-iq(x-y)} \big|_{x_0=t \atop y_0=t}
	=\int\frac{\de^4q}{(2\pi)^3}\,q^0\rho(q)e^{-iq(x-y)} \big|_{x_0=t\atop y_0=t}=\\
	&=\int\frac{\de^4q}{(2\pi)^3}\,q^0\rho(q)e^{i\vec q\cdot(\vec x-\vec y)}
\end{eq}
then comparing the left and the right side of the previous identity one gets exactly eq.~\eqref{eq:sum_rule_spect_func}.

Moreover using eq.~\eqref{eq:rho_sigma} one can rewrite
\begin{eq}\label{eq:sum_rule_sigma}
	1=\int_{-\infty}^{+\infty}\de q^0\,q^0\rho(q)
	=\int_{-\infty}^{+\infty}\de q^0\,|q^0|\sigma(q^2)
	=2\int_0^{+\infty}\de q^0\,q^0\sigma(q^2)
	=\int_0^{+\infty}\de m^2\,\sigma(m^2)
\end{eq}
where using $m^2=q^2=(q^0)^2-\vec q^2$ and the fact that in our integration $\vec q$ is fixed, we applied the change of variable $q^0\mapsto m^2$, $2q^0\de q^0\mapsto \de m^2$. 

Recalling that $\sigma(q^2)\geq0$, we have that $q^0\rho(q)=|q^0|\sigma(q^2)\geq0$ too, hence using eq.~\eqref{eq:sum_rule_spect_func} and eq.~\eqref{eq:sum_rule_sigma} we get that both
\begin{eq}
	A(\omega,\vec q):= \omega\rho(\omega,\vec q) = q^0\rho(q)
\end{eq}
and $\sigma(m^2)$ are probability densities. In particular can be proved that $\sigma(m^2)$ is the probability density to find the state $\int e^{iqx}\ophi(x)\ket0$ in a state of mass $m^2$, while $A(\omega,\vec q)$ is the probability density to find the same state $\int e^{iqx}\ophi(x)\ket0$ in a state of energy $\omega$ for fixed value $\vec q$. 

We just mention that for fermions the same sum rule eq.~\eqref{eq:sum_rule_spect_func} holds equivalently provided that we have the CAR $\{\ophi(\vec x,t),\dot\ophi(\vec y,t)\}=i\delta(\vec x-\vec y)$. 

Denoting by $\Delta_+(x-y;m)$ the 2-points function of the free scalar field of mass, then inserting the equality $\int_0^{+\infty}\de m^2\,\delta(q^2-m^2)=1$ we get
\begin{eq}
	\bra0\phi(x)\phi(y)\ket0
	&=\int\frac{\de^4p}{(2\pi)^3}\sigma(q^2)\theta(q^0)e^{-iq(x-y)}+|\bra0\ophi(0)\ket0|^2\\
	&=\int_0^{+\infty}\de m^2\,\sigma(m^2)\int\frac{\de^4p}{(2\pi)^3}\delta(q^2-m^2)\theta(q^0)e^{-iq(x-y)}+|\bra0\ophi(0)\ket0|^2\\
	&=\int_0^{+\infty}\de m^2\,\sigma(m^2)\Delta_+(x-y;m)+|\bra0\ophi(0)\ket0|^2
\end{eq}
where $|\bra0\ophi(0)\ket0|^2$ is an irrelevant constant which can be removed just shifting $\ophi(x)$. This means that even in the interacting theory we can write the 2-points function as an integral of free 2-points functions of the free field with varying mass weighted by a suitable measure $\de m^2\,\sigma(m^2)$. This fact is a purely relativistic effect, and has no analogue for NRQFT, due to Lorentz invariance, and is no more true if we broke Lorentz symmetry. 

Representation of correlation functions given by eq.~\eqref{eq:2-point-func-spec-repr} and eq.~\eqref{eq:2-point-func-spec-repr-comm} are called \emph{Källen-Lehmann representations} since they was firstly derived by these two physicist.\footnote{Källen: \url{https://doi.org/10.5169\%2Fseals-112316}, Lehmann: \url{https://doi.org/10.1007\%2Fbf02783624}.} Actually Lehmann applied such representation also to the non-relativistic case, as we now show.

\subsubsection{Lehmann representation (non-relativistic case)}

Let's see how to generalize the previous representation without assuming Lorentz invariance. In this case in the Hilbert space $\hs$ of our theory we need:
\begin{enumerate}[label=(\arabic*')]
	\item A unitary representation $U$ of spacetime translations (one can also add rotational invariance).
	\item The field $\ophi$ should transform under an irreducible representation of transformations under $U$.\footnote{For lattice theories $U$ is restricted to discrete spatial lattice translations and typically the time lattice translations are represented contractively, i.e. through terms of the form $e^{-tH}$ (continuum variation of the transfer matrix).} We assume that $\ophi$ transforms as a scalar. 
\end{enumerate}

%Lezione 13/10-2 16.37



\end{document}