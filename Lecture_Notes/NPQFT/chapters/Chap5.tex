\documentclass[../main/main.tex]{subfiles}
\begin{document}

%%%%%%%%%%%%%%%%%%%%%%%
%%%%%%%% LECTURE 11 %%%%%%%%
%%%%%%%%%%%%%%%%%%%%%%%

\chapter{Quantum solitons}

\section{Introduction to topological objects}

The first appearance of a classical \emph{soliton} was in a report by J. Scott Russel in 1842, he was in a boat in a channel and noticed ``a solitary elevation, a rounded, smooth and well-defined heap of water which continued in course along the channel apparently without change of form (i.e. without dispersion/dissipation) or diminution of speed (i.e. constant velocity). 

The term ``soliton'' (which comes from ``solitary wave'') was much later introduced to characterize solutions of wave equations that do not dispense and preserve their form during the propagation. Hence in a sense a classical solution behaves as a particle in spite of being a wave. 

The staticity of these ``wave'' solitons is due to some conservation law. If this conservation law is of topological origin (i.e. the conserved quantity is not related to the (first) Noether theorem and the conservation holds without using all the equations of motion\footnote{For instance, the conservation of the EM current $J_\nu=\partial^\mu F_{\mu\nu}$ does not follow from the equation of motion $\der{p^\mu}{t}=F^{\mu\nu}u_\nu$, but is just a consequence of the antisymmetry of $F_{\mu\nu}$: $\partial^\nu J_\nu=\partial^\mu\partial^\nu F_{\mu\nu}=\partial^{\{\mu}\partial^{\nu\}}F_{[\mu\nu]}=0$.}) these solitons are called \emph{topological}. 

The original concept of topologically protected ``wave'' solitons has been later extended, in particular by high-energy physicists, and today we define a classical topological soliton as a topologically stable, finite energy solution of the Hamiltonian equations of motion of a classical field theory. 

Since already classically solutions behaves as particles, one can naturally expect, correctly, that their quantized version provides a quantum particle, but as we will see later it is a peculiar one. 

\subsection{Spontaneously broken symmetry}

\cite[Chapter C.1]{Strocchi_1985}, \cite{Strocchi:2012}\\

Before entering in the discussion of solitons, a brief formal discussion of the spontaneously broken symmetry (SSB) phenomenon is required. 

We already know that the \emph{observables} of a system are the quantities of the physical system that we can measure and the algebra that they generate is called the \emph{observable algebra} $\obsalg$. The \emph{states} contain the information on the system, if the information is maximal they are called \emph{pure}, if it is not maximal are said \emph{mixed}. 

By definition pure states can be obtained one from the other by operations physically performed on the system (including limiting procedure, such as the infinite volume limit or the thermodynamic limit, but excluding transformations which require to cross would-be states of infinite ). Both in the classical and quantum settings the mixed states can be viewed as complex combinations of pure states and a state is pure if it cannot be written as convex combination of the pure states. 
For equilibrium states at finite temperature by analogy we introduce the concept of \emph{pure phase}: an equilibrium state is a pure phase if it cannot be written as a convex combination of other equilibrium states.

\skipline

For instance consider the Ising model: take a finite lattice of volume $V$, where to each cell $i$ is associated a classical spin $\sigma_i$ with possible values $\pm1$. Suppose that we imposed vanishing boundary conditions, i.e. outside the volume we do not have any spin (the system is confined), and try to take the infinite volume limit. For spatial dimension $d=2$ and temperature $T<T_c$ where $T_c$ is the \emph{critical temperature} the system has two possible ground states: for the first one $\langle\sigma_i\rangle>0$ for all the cells, for the second one $\langle\sigma_i\rangle<0$. Obviously, the equilibrium state for the system in such condition is described by $\langle\sigma_i\rangle =0$ for all the possible cells, and it is not a pure phase as it can be described as a convex combination of the two ground states, which in turn are (unstable) equilibrium states. 

\skipline

We can now distinguish different kinds of symmetries. An \emph{algebraic symmetry} is an invertible algebra homomorphism $\obsalg\to\obsalg$ (i.e. a map preserving the algebraic relations, e.g.  equations of motion are preserved in form and also (in quantum setting) the commutation relations between observables). A \emph{physical symmetry} is an algebraic symmetry together with an invertible map $\purstate\to\purstate$, where $\purstate$ is the space of pure states, which preserves the expectation values:
\begin{eq}
	\begin{aligned} 
		\alpha:\obsalg&\longrightarrow\obsalg\\
		A&\longmapsto A'
	\end{aligned}
	\qquad\text{and}\qquad
	\begin{aligned}
		\tilde \alpha:\purstate&\longrightarrow\purstate\\
		\Sigma&\longmapsto\Sigma'
	\end{aligned}
\end{eq}
such that
\begin{eq}
	\langle A\rangle_\Sigma=\langle A'\rangle_{\Sigma'}
\end{eq}
A \emph{dynamical symmetry} is either an algebraic or a physical symmetry leaving the Hamiltonian invariant
\begin{eq}
	H\longmapsto H'=H
\end{eq}

A dynamical symmetry is said \emph{spontaneously broken} if it cannot be realized as a physical symmetry, i.e. is an algebraic symmetry leaving $H$ invariant but it does no exists a map in the space of pure states of the system preserving the expectation values. 

\skipline

Naively it seems impossible to define a spontaneously broken symmetry, as for each map $\alpha:\obsalg\to\obsalg$ always exists a map $\tilde\alpha:\purstate\to\purstate$ defined by
\begin{eq}
	\langle \alpha^{-1}(A)\rangle_\Sigma=\langle A\rangle_{\tilde\alpha(\Sigma)}
\end{eq}
which clearly implies
\begin{eq}
	\langle A\rangle_\Sigma=\langle\alpha(A)\rangle_{\tilde\alpha(\Sigma)}=\langle A'\rangle_{\Sigma'}
\end{eq}
What goes wrong is that such map $\tilde\alpha:\Sigma\mapsto\Sigma'$ in principle may send the space $\purstate$ not in itself, i.e. in principle $\tilde\alpha(\Sigma)\not\in\purstate$, but in another space of states not realizable by means of physical operations on the system starting from the initial states in $\purstate$.

\skipline

Take for instance the Heisenberg model of a ferromagnet in three dimension, where the spin is classically described by a unit vector in a lattice. Let's then consider the thermodynamic limit. It is well known that such system has a critical temperature $T_c$, such that for $T<T_c$ the ground states have an expectation value $\langle\vec s_i\rangle\neq0$. For $T=0$ all the spin are aligned in the same direction. The Hamiltonian for this system is 
\begin{eq}
	H=J\sum_{\langle i,j\rangle}(\vec s_i-\vec s_j)^2
\end{eq} 
which is clearly invariant under rotations. Consider two ground states at $T=0$ which differs for the orientation of the spins, and let's see if these can be related by some allowed transformation or not, if not we would have a spontaneous symmetry breaking. We are allowed to do only physical allowed (in particular, local) transformations, at least by some limiting procedure. Nevertheless we already took the thermodynamic limit, therefore we are not allowed to move all the spins (including those at infinity) at the same time with a single transformation. We are forced to use some limiting procedure, for instance we can consider neighbourhoods of circles of radius $R$ and smoothly rotate the spins in these regions, while we increase $R$ from 0 to $\infty$. But as $R$ approaches $\infty$ there are infinitely many spins rotated simultaneously, hence due to the non trivial interacting energy between spins, one can expect that such procedure requires an infinite amount of energy. If the explicit computation of the energy actually give us an infinite energy cost, then spontaneous symmetry breaking occured. 

Notice that in general one cannot guess whether we have SSB or not, a computation of the energy (or some other relevant physical quantity) is needed. 

\skipline

Unfortunately sometimes one refers to SSB in the case in which the Hamiltonian is invariant under a symmetry $\alpha$, but each of its pure equilibrium states are not invariant under $\tilde\alpha$, hence restricting the non-invariance to equilibrium states. 

For field theories are basically equivalent from the practical point of view but for systems with finite degrees of freedom they are not. For instance consider a horizontal place where gravitational force applies in the vertical direction, and put a point particle at rest a point, then certainly this describes an equilibrium state. The Hamiltonian is invariant under horizontal space translations, nevertheless the equilibrium state is not, hence according to the latter definition the symmetry is spontaneously broken. On the other side, certainly this is not a SSB according to the first definition, because we can easily move to another equilibrium state by some physically allowed translation, since all the intermediate states are pure states of the system. 

\section{Kinks}

\cite[Chapter 2]{Shifman:2012}\\

The fist (topological) soliton that we consider is the \emph{kink} in $\phi^4_2$, i.e. the relativistic $\phi^4$ theory in 1+1 dimensions in the spontaneously broken phase. The classical Hamiltonian we consider is on the real line $\R$, labelled by $x$, and if the system is described by a real scalar field $\phi(x,t)$ with conjugated momentum $\pi(x,t)$ (the dependence on time of the fields is understood) it reads
\begin{eq}
	H=\int\de x\,\ham\big(\pi,\phi\big)=\int\de x\,\left[\half\pi^2+\half\left(\der{\phi} x\right)^2+\frac{g^2}4\big(\phi^2-v^2\big)^2\right]
\end{eq}
with $v,g\in\R$. This potential is the field analogue of the double well with potential $\frac{g^2}4(x^2-v^2)^2$.
The equations of motion in Hamiltonian formalism are
\begin{eq}
	\dphi&=\fder{H}{\pi}=\pder{\ham}{\pi}=\pi\\
	\dpi&=\fder{H}{\phi}=\pder{\ham}{\phi}=-\dpder{}{x}\phi+g^2\big(\phi^2-v^2\big)\phi
\end{eq}
where for the second equation we used an appropriate integration by parts in $H$. 
Equilibrium states are given by the equilibrium condition $\dphi=\dpi=0$, which implies (for each $x\in\R$)
\begin{eq}
	\pi=0
	\tand
	\dpder{}{x}\phi=g^2\big(\phi^2-v^2\big)\phi
\end{eq}
Clearly, from the structure of the Hamiltonian, finite energy solutions must satisfy 
\begin{eq}\label{eq:kink-boundary-conditi}
	\pi\xrightarrow[x\to\pm\infty]{}0
	\tand
	\phi\xrightarrow[x\to\pm\infty]{}\pm v
\end{eq}
as otherwise $H$ is divergent. Indeed all the terms inside $H$ are strictly positive, hence both the term depending on $\pi$ and the term depending on $\phi$ should vanish identically at infinite. The absolute minima of $H$ are given by the two configurations
\begin{eq}
	\phi_+^0=+v \tcomma \pi^0=0
	\tand
	\phi_-^0=-v \tcomma \pi^0=0
\end{eq}
In such cases $H$ vanishes, hence both these solutions are called \emph{vacuum}. And the expectation values of the field at the minima are
\begin{eq}
	\langle\phi\rangle_{\phi_\pm^0,\pi^0}=\pm v
\end{eq}

\skipline

The Hamiltonian is invariant under the symmetry $\alpha$
\begin{eq}\label{eq:symm_kink}
	\phi\mapsto-\phi
	\tcomma
	\pi\mapsto-\pi
\end{eq}
We see that in order to satisfy
\begin{eq}
	\langle\phi\rangle_\Sigma=\langle\alpha(\phi)\rangle_{\tilde\alpha(\Sigma)}
\end{eq}
for $\Sigma=(\phi^0_\pm,\pi^0)$ we should have 
\begin{eq}
	\pm v=\langle-\phi\rangle_{\tilde\alpha(\phi^0_\pm, \pi^0)}
\end{eq}
which is satisfied for $\tilde\alpha(\phi^0_\pm,\pi^0)=(\phi^0_\mp,\pi^0)$.
However it is impossible to reach continuously $\phi_-^0$ from $\phi_+^0$ and vice versa, since the set of boundary conditions at infinity, $\{+v,-v\}$, is not connected and to reach one minimum from the other with (the limit of) a local transformation we need to cross infinite energy states. Therefore symmetry eq.~\eqref{eq:symm_kink} is spontaneously broken, and the two minima that we found lives in disjoint space of states related by a parity transformation.

\subsubsection{First-order equation}

However there are two local minima in the space of finite-energy solutions: we can find them with the following trick. For $\pi=\dphi=0$ the equation of motion reduces to
\begin{eq}\label{eq:eom-kink}
	\der{^2}{x^2}\phi=\frac{g^2}4(\phi^2-v^2)\phi
\end{eq}
where now $\phi$ depends only on $x$. If we think about $x$ as a time coordinate and $\phi$ as the position $q$  of a particle, then eq.~\eqref{eq:eom-kink} become a Newton equation with potential $V(q)=\frac{g^2}4(q^2-v^2)^2$. Due to energy conservation we get
\begin{eq}
	\half\dot q^2-\frac{g^2}4(q^2-v^2)^2=\tconst=0
\end{eq}
where the vanishing of the energy is due to the boundary condition eq.~\eqref{eq:kink-boundary-conditi}. Hence we can solve the previous differential equation:
\begin{eq}
	\der qt=\pm\frac g{\sqrt2}(q^2-v^2)
\end{eq}
i.e.
\begin{eq}
	\der\phi x=\pm\frac g{\sqrt2}(\phi^2-v^2)
\end{eq}
These are standard first order equations trivially solved by 
\begin{eq}
	\int\de x=\mp\frac{\sqrt2}{g}\int\frac{\de\phi}{\phi^2-v^2}
\end{eq}
which have solutions
\begin{eq}
	x=\pm\frac{\sqrt2}{gv}\arctanh\left(\frac\phi v\right)+x_0
\end{eq}
i.e.
\begin{eq}
	\phi_\pm^S(x)=\pm v\tanh\left(\frac{gv}{\sqrt2}(x-x_0)\right)
\end{eq}
where we denoted the two solutions by $S$, which means ``soliton''. In fig.~\ref{fig:kink-solution} one can find a plot of the function $\phi_+^S(x)$. The value $x_0$ is called \emph{collective coordinate} or the \emph{modulus} of the soliton, called (classical) \emph{kink} ($\phi_+^S(x)$) or (classical) \emph{anti-kink} ($\phi_-^S(x)$).

\begin{figure}[h]
\centering
\begin{tikzpicture}
  \draw[->] (-4, 0) -- (5, 0) node[at={(5.1,0.25)}] {$x$};
  \draw[->] (0, -2.2) -- (0, 2.5) node[at={(0.7,2.6)}] {$\phi_+^S(x)$};
  \draw[scale=0.5, domain=-7:9, smooth, variable=\x, blue, very thick] plot ({\x}, {3*tanh(0.5*(\x-1))});
  \draw[scale=0.5, domain=-7:9, smooth, variable=\x, red] plot ({\x}, {3});
  \draw[scale=0.5, domain=-7:9, smooth, variable=\x, red] plot ({\x}, {-3});
  \draw (4.5,1.8) node {$+v$};
  \draw (4.5,-1.25) node {$-v$};
  \draw (0.7,-0.2) node {$x_0$};
\end{tikzpicture}
\caption{Shape of the solution $\phi_+^S(x)$.} %parameters: x_0=1, v=3, g=1/3sqrt2 
\label{fig:kink-solution}
\end{figure}

We see that $\phi_+^S$, $\phi_-^S$, $\phi_+^0$ and $\phi_-^0$ cannot be deformed one into other by operations physically implementable, since the boundary conditions at $\infty$ are disjoint. This statement can be translated for the solitons in the conservation of a charge
\begin{eq}\label{eq:cons-charge-kink}
	Q=\int\de x\der{}{x}\phi=\phi(+\infty)-\phi(-\infty)
\end{eq}
which is topological (only the behaviour at infinity is relevant). The corresponding conserved current is 
\begin{eq}
	J_\mu(x)=\lctens_{\mu\nu}\partial^\nu\phi
\end{eq}
with $\mu,\nu=0,1$ where $\mu=0$ is the time $t$ component and $\mu=1$ is the spatial $x$ component, moreover $\lctens$ is the Levi-Civita tensor. We can easily see that $J_0$ actually coincides with the conserved density associated to \eqref{eq:cons-charge-kink}:
\begin{eq}
	\int\de x\, J_0=\int\de x \,\lctens_{0\nu}\partial^\nu\phi=
	\int\de x \,\partial^1\phi=\int\de x\,\pder{}x\phi=Q
\end{eq}
while $J_1$ is the unique component of the vector current. The current $J_\mu$ is automatically conserved without using the equations of motion, since the conservation directly follows from the contraction of a symmetric tensor with an antisymmetric one
\begin{eq}
	\partial^\mu J_\mu=\lctens_{\mu\nu}\partial^\mu\partial^\nu\phi=\lctens_{[\mu\nu]}\partial^{\{\mu}\partial^{\nu\}}\phi=0
\end{eq}

%%%%%%%%%%%%%%%%%%%%%%%
%%%%%%%% LECTURE 12 %%%%%%%%
%%%%%%%%%%%%%%%%%%%%%%%

\subsubsection{The Bogomol'nyi bound}

There is a more instructive way to prove that $\phi_\pm^S$ are the local minima at $(\mp v,\pm v)$ boundary conditions. Let's write the potential $V(\phi)$ in terms of a function $\suppot(\phi)$ called \emph{superpotential} (since it is largely used in supersymmetric theories) by
\begin{eq}
	V(\phi)=\half\left(\der\suppot\phi\right)^2
\end{eq}
Since $V(\phi)=\frac{g^2}4(\phi^2-v^2)^2$ this implies that
\begin{eq}
	\der\suppot\phi=\frac g{\sqrt2}(\phi^2-v^2)
	\tand
	\suppot=\frac g{\sqrt2}\left(\frac13\phi^3-v^2\phi\right)
\end{eq}
The Hamiltonian can be rewritten for $\pi=0$ as
\begin{eq}\label{eq:Ham-kink-suppot}
	H=\int\de x\,\left[\half\left(\der\phi x\pm\der\suppot\phi\right)^2\mp\der\phi x\der\suppot\phi\right]
	\quad\text{for boundary conditions }(\mp v,\pm v)
\end{eq}
Now notice that the last term gives
\begin{eq}
	&\int\de x\,\der\suppot\phi\der\phi x=\suppot(\phi(+\infty))-\suppot(\phi(-\infty)
	=\pm\Delta\suppot\quad\text{for boundary conditions }(\mp v,\pm v)
\end{eq}
with
\begin{eq}
	\Delta\suppot=\suppot(v)-\suppot(-v)=-\frac43\frac g{\sqrt2}<0
\end{eq}
hence we get that
\begin{eq}\label{eq:Bogom-bound}
	H\geq-\Delta\suppot>0
\end{eq}
for both the possible boundary conditions. This lower bound  is called \emph{Bogomol'nyi bound} (as we will see has several generalizations). This bound is saturated, hence the equality in eq.~\eqref{eq:Bogom-bound} holds, only if the term inside square brackets in eq.~\eqref{eq:Ham-kink-suppot} vanishes, i.e. if the following first order equation is satisfied
\begin{eq}
	\der\phi x=\pm\der\suppot\phi
\end{eq}
which coincides exactly with eq.~\eqref{eq:eom-kink}. Hence the solutions $\phi_\pm^S$ are local minima for which the energy coincides with the Bogomol'nyi bound. 

\skipline

The energy density associated to $\phi_\pm^S(x_0)$, depending on the modulus $x_0$, is given by the Hamiltonian density evaluated for the field $\phi_\pm^S(x_0)$ (it obviously depends on the spatial coordinate $x$ and on the choice of $x_0$):
\begin{eq}
	\ham&=\half\left(\der{\phi_\pm^S}x\right)^2+\frac{g^2}4\left(\phi_\pm^{S\,2}-v^2\right)^2
	=\half\left(\der{\phi_\pm^S}x\right)^2+\half\left(\der{\suppot(\phi_\pm^S)}x\right)^2
	=\frac{g^2v^4}2\sech^4\left(\frac{gv}{\sqrt2}(x-x_0)\right)
\end{eq}
In fig.~\ref{fig:kink-energy} one can see the shape of $\cenergy(x)$ associated to the field of fig.~\ref{fig:kink-solution}. It is clearly localized near $x_0$. Hence the soliton sxhibits an energy density profile as a `` dump'' around $x_0$, thus behaving ``like a particle'' at $x_0$. The integral 
\begin{eq}
	\int\de x\,\cenergy(x)=-\Delta\suppot=\frac4{3\sqrt2}gv^3
\end{eq}
can be interpreted as the ``classical'' mass of the soliton $M_S^\tcl$, i.e. the energy of the soliton in its rest frame. 

\begin{figure}[h]
\centering
\begin{tikzpicture}
  \draw[->] (-4, 0) -- (5, 0) node[at={(5.1,0.25)}] {$x$};
  \draw[->] (0, -1) -- (0, 3) node[at={(0.5,3.1)}] {$\cenergy(x)$};
  \draw[scale=0.5, domain=-7:9, smooth, variable=\x, blue, very thick] plot ({\x}, {4.5*(cosh((\x-1)/2))^(-4)});
   \draw[scale=0.5, domain=0:5, smooth, variable=\y, dash pattern=on 3pt off 3pt] plot ({1}, {\y});
  \draw (0.5,-0.2) node {$x_0$};
\end{tikzpicture}
\caption{Shape of the energy density for the field $\phi_+^S(x)$}{plotted in fig.~\ref{fig:kink-solution} (the $x$-axis has not be rescaled).} %parameters: x_0=1, v=3, g=1/3sqrt2 
\label{fig:kink-energy}
\end{figure}

Since (looking at the Hamiltonian) the theory is Lorentz invariant (on 1+1 dimension) one can ``change the reference frame'' and make the soliton move at a constant velocity $V$:
\begin{eq}
	\phi_\pm^S(x,x_0,V)=\pm v\tanh\left(\frac{gv}{\sqrt2}\,\frac{x-x_0-Vt}{\sqrt{1-V^2}}\right)
\end{eq}
i.e. behaves like a particle with equation of motion $x=x_0+Vt$.

\subsubsection{Semi-classical description}

Up to now we just described the soliton at classical level. Let us turn to the quantum world. We first give a standard semi-classical (heuristic) Hamiltonian treatment, then we make some comments on a rigorous approach in the spirit of the reconstruction theorem previously discussed. 

Let us start from the quantization of fluctuations around one vacuum. To quantize perturbatively fluctuations around $\phi_+^0=v$ we rewrite the quantum field as
\begin{eq}
	\ophi(x)=v+\ochi(x)
\end{eq}
Formally (in principle regularizations are needed) expanding in $\ochi$ in the Hamiltonian we get
\begin{eq}	
	H&=\int\de x\,\left[\frac12\opi^2+\frac12\left(\der\ochi x\right)^2+\frac{g^2}4\left(2v\ochi+\ochi^2\right)^2\right]\\
	&=\int\de x\,\left[\frac12\opi+\frac12\left(\der\ochi x\right)^2+\frac{2(gv)^2}2\ochi^2+\polyn(\ochi)\right]
\end{eq}
where $\polyn(\ochi)$ is a polynomial in $\ochi$ of order higher than 2. Integrating by part we get
\begin{eq}
	H_2=\int\de x\,\left[\frac12\opi+\frac12\ochi\left(-\der{^2}{x^2}+2(gv)^2\right)\ochi\right]
\end{eq}
($H_2$ denotes the quadratic part on $H$, describing the free field). One can then expand $\ochi$ in terms of the complete set of solutions of the differential equation
\begin{eq}
	\left(-\der{^2}{x^2}+2(gv)^2\right)\chi_k=\omega^2(k)\chi_k
\end{eq}
which we know are just the complex exponentials $e^{ikx}$ with eigenvalues
\begin{eq}
	\omega^2(k)=k^2+2(gv)^2
\end{eq}
and then one can write
\begin{eq}
	\ochi(x)=\sum_k\op a(k)\chi_k(x)
\end{eq}
From the quadratic terms we see that $\ochi$ is a massive field with (bare) mass $m=\sqrt2gv$, such that we recover the usual dispersion relation $\omega(k)=\sqrt{k^2+m^2}$, and polynomial interaction given by $\polyn(\ochi)$.

%Lecture 10/11-1 min 42.13 pag 19




%%%%%%%%%%%%%%%%%%%%%%%
%%%%%%%% LECTURE 13 %%%%%%%%
%%%%%%%%%%%%%%%%%%%%%%%

\end{document}