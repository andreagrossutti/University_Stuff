\usepackage[T1]{fontenc}
\usepackage[utf8]{inputenc}
\usepackage[english]{babel}
\usepackage{geometry}
	\geometry{a4paper, top=3cm, bottom=3cm, left=2.5cm, right=2.5cm}
\usepackage{amsfonts}
\usepackage{amsmath}
\usepackage{latexsym}
\usepackage{cases}
\usepackage{mathtools}
\usepackage{slashed}
\usepackage{physics}
\usepackage{mathrsfs}
\usepackage{quoting}
\usepackage{bbold} %identity symbol (double 1)
\usepackage[framemethod=TikZ, nobreak=true]{mdframed}
\usepackage{amsthm}
\usepackage[compat=1.1.0]{tikz-feynman}
\usepackage{enumitem}
	\setlist{nosep}
	\setlist{label=(\roman*)}
\usepackage[framemethod=TikZ]{mdframed}%ambient thm ecc...
	\usepackage{amsthm}
\usepackage{booktabs}
\usepackage{caption}
\usepackage{dsfont}
\usepackage{float}
\usepackage{tikz}
\usepackage{mathdots}
\usepackage{yhmath}
\usepackage[thicklines]{cancel}
	\renewcommand{\CancelColor}{\color{lightgray}}
\usepackage{color}
\usepackage{siunitx}
\usepackage{array}
\usepackage{multirow}
\usepackage{amssymb}
\usepackage{gensymb}
\usepackage{tabularx}
\usepackage{booktabs}
\usetikzlibrary{fadings}
\usetikzlibrary{patterns}
\usetikzlibrary{shadows.blur}
\usepackage{hyperref}
\hypersetup{
	linktoc=all,
	colorlinks=true,
	linkcolor={blue!70!black},
	citecolor={green}
}
%\usepackage[colorinlistoftodos,textsize=tiny]{todonotes}% TO DO points


%footnote con numeri romani invece dei numeri arabi
\renewcommand{\thefootnote}
	{\Roman{footnote}}
	
\newenvironment{eq}{\begin{equation}\begin{aligned}}{\end{aligned}\end{equation}\ignorespacesafterend}
\newenvironment{eq*}{\begin{equation*}\begin{aligned}}{\end{aligned}\end{equation*}\ignorespacesafterend}


%scambia phi e varphi
%\let\temp\phi
%\let\phi\varphi
%\let\varphi\temp

%scambia epsilon e varepsilon
\let\temp\epsilon
\let\epsilon\varepsilon
\let\varepsilon\temp

\let\vect\vec
\let\vec\boldsymbol

\newcommand{\half}{\frac12}
\newcommand{\p}[1]{\left(#1\right)}
\newcommand{\N}{\mathbb{N}} %Naturali
\newcommand{\Z}{\mathbb{Z}} %Interi
\newcommand{\R}{\mathbb{R}} %Reali
\newcommand{\C}{\mathbb{C}} %Complessi

\newcommand{\loren}{L_+^\uparrow} %restricted Lorentz group
\newcommand{\poinc}{\mathcal P_+^\uparrow} %restricted Poincaré group
\newcommand{\cpoinc}{\tilde {\mathcal P}_+^\uparrow} %covering of restricted Poincaré group

\renewcommand{\to}{\rightarrow}
\newcommand{\ot}{\leftarrow}
\newcommand{\so}{\Rightarrow}
\newcommand{\os}{\Leftarrow}

\newcommand{\id}{\mathbb 1} %Identità con il doppio 1
\newcommand{\varid}{\textup {id}} %Identità con "id"

\newcommand{\de}{\text{d}} %Differential
\newcommand{\der}[2]{\frac{\de #1}{\de #2}}
\newcommand{\dder}[3]{\frac{\de^2 #1}{\de #2\,\de #3}}
\newcommand{\pder}[2]{\frac{\partial #1}{\partial #2}}
\newcommand{\pdder}[3]{\frac{\partial^2 #1}{\partial #2\,\partial #3}}
\newcommand{\dpder}[2]{\frac{\partial^2 #1}{{\partial #2}^2}}

\newcommand{\inv}[1]{{#1}^{-1}}
\newcommand{\sign}{\operatorname{sign}}

\newcommand{\hs}{\mathcal H} %Hilbert space
\newcommand{\fock}{\mathcal F} %Fock space
\newcommand{\dom}{D} %domain of an operator
\newcommand{\contour}{\mathcal C} %contour integral
\newcommand{\lag}{\mathcal L} %lagrangian density
\newcommand{\ham}{\mathcal H} %hamiltonian density
\newcommand{\fourier}{\mathcal F} %Fourier transform

\newcommand{\cinf}{C^\infty} %infinitely derivable functions

\newcommand{\cenergy}{\mathcal E} %curly energy simbol
\newcommand{\efield}{\textbf E} %electric field

\newcommand{\modx}{|\vec x|}
\newcommand{\modk}{|\vec k|}
\newcommand{\modp}{|\vec p|}
\newcommand{\modpp}{|\vec{p'}|}

\renewcommand{\op}[1]{\hat{#1}}%\operator
\newcommand{\dotop}[1]{\dot{\op{#1}}} %time derivative of operator
\newcommand{\supp}{\operatorname{supp}} %support of an operator
\newcommand{\spec}{\sigma} %spectrum
\newcommand{\st}{\,|\,} %such that
\newcommand{\varst}{\,\text{s.t.}\,} %var such that
\newcommand{\tp}{\text T}%time ordering product

\newcommand{\ophi}{\op\phi} %field operator phi
\newcommand{\ophid}{\op\phi^\dagger} %adjoint of field operator phi
\newcommand{\opsi}{\op\psi} %field operator psi
\newcommand{\opi}{\op\pi} %field operator pi

\newcommand{\retc}{G_{\text{ret}}} %retarded correlator
\newcommand{\fretc}{\tilde G_{\text{ret}}} %fourier transform of retarded correlator
\newcommand{\mret}{G_{\text{M}}} %Matsubara correlator

\newcommand{\uei}{U_\epsilon^I} %unitary operator in interaction picture for finite epsilon
\newcommand{\ueid}{U_\epsilon^{I\dagger}} % adj unitary operator in interaction picture for finite epsilon
\newcommand{\scate}{S_\epsilon}

\newcommand{\dir}{\oplus} %direct sum
\newcommand{\bigdir}{\bigoplus} % big direct sum
\newcommand{\ten}{\otimes} %tensor product

\newcommand{\sls}{\,{\mathbin{\tikz [x=1.4ex,y=1.6ex,line width=.2ex] \draw (0,0) -- (1,1) (0,1) -- (1,0);}}\,} %space-like separated coordinates

\newcommand{\tmax}{\text{max}}
\newcommand{\trad}{\text{rad}}
\newcommand{\tin}{\text{in}}
\newcommand{\tout}{\text{out}}
\newcommand{\tinc}{\text{inc}}

\newcommand{\tcomma}{\quad,\quad}
\newcommand{\twith}{\quad\text{with}\quad}
\newcommand{\tand}{\quad\text{and}\quad}
\newcommand{\tif}{\quad\text{if}\quad}
\newcommand{\tfor}{\quad\text{for}\quad}
\newcommand{\tforall}{\quad\text{for all}\quad}
\newcommand{\tforany}{\quad\text{for any}\quad}

\newcommand{\skipline}{\leavevmode\vspace{\baselineskip}} %Salta una riga (devi saltare una riga prima di mettere il comando)

\newcommand*\circled[1]{\tikz[baseline=(char.base)]{
            \node[shape=circle,draw,inner sep=2pt] (char) {#1};}}%numeri cerchiati

%Struttura modulare
\usepackage{subfiles}
\newcommand{\onlyinsubfile}[1]{#1}
\newcommand{\onlyinmainfile}[1]{}

\definecolor{blue}{RGB}{0, 114, 178} 
\definecolor{red}{RGB}{255, 51,0}
\definecolor{green}{RGB}{0, 153, 0}

%Ambienti riquadrati
% nel file .tex: \begin{exercise}[TITOLO ESERICZIO], se non vuoi dare un titolo all'esercizio non mettere le parentesi quadre (lo stesso vale per example)

\mdfdefinestyle{example}{%
linecolor=gray,linewidth=2pt,%
frametitlerule=true,%
frametitlebackgroundcolor=gray!20,
innertopmargin=\topskip,
}
\mdtheorem[style=example, nobreak=false]{example}{Example}

\mdfdefinestyle{exercize}{%
linecolor=blue,linewidth=2pt,%	
frametitlerule=true,%
frametitlebackgroundcolor=blue!20,	% !20 = saturazione del colore
innertopmargin=\topskip,
}
\mdtheorem[style=exercize, nobreak=false]{exercize}{Exercize}

\mdfdefinestyle{todo}{%
linecolor=red,linewidth=2pt,%	
frametitlerule=true,%
frametitlebackgroundcolor=red!20,% !20 = color saturation
innertopmargin=\topskip,
}
\mdtheorem[style=todo, nobreak=false]{todo}{TO DO} %use \begin{todo} ... \end{todo}

\mdfdefinestyle{tocheck}{%
linecolor=green,linewidth=2pt,%	
frametitlerule=true,%
frametitlebackgroundcolor=green!20,% !20 = color saturation
innertopmargin=\topskip,
}
\mdtheorem[style=tocheck, nobreak=false]{tocheck}{TO CHECK} %use \begin{tocheck} ... \end{tocheck}

\mdfdefinestyle{mybox}{
	leftmargin = 1cm,
	rightmargin = 1cm,
	innerleftmargin = 5 pt,
	innerrightmargin = 5 pt,
	font = \itshape\rmfamily,
}
% use \begin{mdframed}[style=mybox] -> \end{mdframed}

	\theoremstyle{plain}%
	\newtheorem{theorem}{Theorem}[chapter]%
	\newtheorem{proposition}[theorem]{Proposition}%
	\newtheorem{corollary}{Corollary}[theorem]%
	\newtheorem{lemma}[theorem]{Lemma}%
	\newtheorem{property}[theorem]{Property}%
	\newtheorem{axiom}{Axiom}%
	\newtheorem{problem}{Problem}%
	\newtheorem{question}{Question}%
	\newtheorem{claim}{Claim}%
	%
	\theoremstyle{definition}%
	\newtheorem{definition}[theorem]{Definition}%
%	\newtheorem{example}{Example}[chapter]%
	\newtheorem{counterexample}[example]{Counterexample}%
	\newtheorem{nonexample}[example]{Non-Example}%
	\newtheorem{impexample}[example]{Important Example}%
	\newtheorem{exercise}[example]{Exercise}%
	\newtheorem{construction}[theorem]{Construction}%
	%
	\theoremstyle{remark}%
	\newtheorem{remark}{Remark}[theorem]%
	%
	% starred environments defs
	\theoremstyle{plain}%
	\newtheorem*{theorem*}{Theorem}%
	\newtheorem*{proposition*}{Proposition}%
	\newtheorem*{corollary*}{Corollary}%
	\newtheorem*{lemma*}{Lemma}%
	\newtheorem*{property*}{Property}%
	\newtheorem*{axiom*}{Axiom}%
	\newtheorem*{problem*}{Problem}%
	\newtheorem*{question*}{Question}%
	\newtheorem*{claim*}{Claim}%
	%
	\theoremstyle{definition}%
	\newtheorem*{definition*}{Definition}%
%	\newtheorem*{example*}{Example}%
	\newtheorem*{counterexample*}{Counterexample}%
	\newtheorem*{nonexample*}{Non-Example}%
	\newtheorem*{impexample*}{Important Example}%
	\newtheorem*{exercise*}{Exercise}%
	\newtheorem*{construction*}{Construction}%
	%
	\theoremstyle{remark}%
	\newtheorem*{remark*}{Remark}%

\usepackage[backend=biber, style=alphabetic, sorting=nty]{biblatex}
\addbibresource{Bibliography.bib}