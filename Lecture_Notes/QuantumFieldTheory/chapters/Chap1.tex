\documentclass[../main/main.tex]{subfiles}



\begin{document}



\chapter{The LSZ Reduction Formula}


\section{A new approach to Quantum Field Theory}
In this chapter we will make a first contact between operator formalism  and path integral formalism of QFT. 
QFT has main objective of compute:
\begin{enumerate}
\item scattering amplitudes;
\item cross-section, where well defined and separated states interacts, creating new final states.
\end{enumerate}
By now, these objective are obtained using canonical quantization formalism. This tool is useful for several situations but, as we will see, there is something more in QFT.

Another point to stress is that the approach we used so far is a perturbative method, which is a really good and precise method, and gave very good results. Nevertheless this method gives an incomplete information. With the standard approach of computing cross section using operator formalism, we missed some informations about the process we are studiying. So is important to have a different tool that allows us to obtain this non perturbative information. This tool is called \textbf{Path Integral formalism}.

With this method we will obtain all results we obtained with the perturbative method, but we will be able to understand better our previous results and also something more about QFT, such as non-perturbative effects. This is important because QFT plays, beside its role in Fundamental Interactions, a fundamental role also in Statistical Mechanics, Cosmology, Topology, Geometry and String Theory, where perturbative approach is not so efficient (or it is useless at all).

First of all we need to understand which are limits of operatorial formalism, and then which is the relation between the two formalism.



\subsubsection{Limits of the Operatorial Formalism}
In the operator formalism approach the step we done are
\begin{enumerate}
\item pick a time variable $t$;
\item define an hamiltonian $H=H(\phi, \pi)$ and the field conjugated field 
\[\pi=\frac{\delta\mathcal L}{\delta\dot\phi(x)}\]
\item transform $\phi$ and $\pi$ these into operators
\item impose that at equal time some (anti)commutation rules between fields are satisfied
\[[\phi(t,\vec x),\pi(t,\vec y)]=i\delta^3(\vec x-\vec y)\]
\end{enumerate}
Pros: 
\begin{enumerate}
\item Clear physical content (states, operators), expecially manifest unitariety (thanks to hamiltonian approach)
\end{enumerate}
Cons:
\begin{enumerate}
\item Covariance not manifest at intermediate stages
\item Derivation of Feynman rules can be complicated.\\
For example, consider a lagrangian with a term that mixes interaction and kinematic terms
\[\mathcal L =-\frac12\partial_\mu\phi\partial^\mu-\lambda\phi^2\partial_\mu\phi\partial^\mu\]
The conjugated field is 
\[\pi=\partial_0\phi(1+2\lambda\phi^2)\]
and the hamiltonian reads
\[H=\pi\partial_0\phi-\mathcal L=\frac12\frac{\pi^2}{1+2\lambda\phi^2}+\dots\]
So we don't have only $\pi^2$ but also an interaction term between $\pi$ and $\phi$. Assume $\lambda\ll1$ then using perturbative approach
\[H=\frac12\pi^2(1-2\lambda\phi^2+4\lambda^2\phi^4+\dots)\]
so we obtain an infinite number of vertex related to terms $\lambda^n\phi^{2n}\pi$.
In this case deriving Feynman rules is a really hard task.
\item  In order to obtain a nice analysis of gauge theories ( in particular non-abelian ones) at levels up to the tree level the perturbative approach is not so good. To obtain proprieties of these theories we must go beyond using use path integral approach.
\end{enumerate}


\section{Correlators and the LSZ reduction formula}

\textbf{Important:} \textit{Accoarding to the notation used in Srednicki, in this section we will use the metric}
\[g^{\mu\nu}=\begin{pmatrix}-1&&&\\&1&&\\&&1&\\&&&1\end{pmatrix}\]


In this section we will obtain the connection between perturbative formalism and path integral formalism.
Recall from previous courses that \emph{correlators} or \emph{Green functions} contains most of informations about QFT
\[G(x_1,x_2,\dots,x_n)=\bra0T[\phi(x_1),\dots,\varphi(x_n)]\ket0\]

\subsubsection{Covariant convention for states}
\textsf{Srednicki, chap 3}.\\

We start with a very simple theory, a free massive scalar field with lagrangian
\[\mathcal L_0=-\frac12\partial_\mu\phi\,\partial^\mu\phi-\frac{m^2}2\phi^2\]
whose  equation of motion is 
\[\square\phi=m^2\phi\]
and normalized solutions are in the form of plane waves\footnote{$x^\mu=(t,\vec x)$ and $k^\mu=(\omega,\vec k)$. Depending on the contest, when $k^0$ is not used as integration variable we assume $k^0=\omega$.}
\[ \frac1{\sqrt{(2\pi)^32k^0}}e^{\pm ik x}\qquad\textup{where}\quad k^2=\vec k^2-\omega^2=-m^2\]
I can write scalar field in terms of $\alpha$ and $\alpha^\dagger$ operators:\footnote{If I consider a real field $\alpha(x)=\alpha^\dagger(x)$}
\[\phi(x)=\int\frac{\de^3\vec k}{\sqrt{(2\pi)^32k^0}}\left(\alpha(\vec k)e^{ikx}+\alpha^\dagger(\vec k)e^{-ikx}\right)\]
When we quantize the theory we introduce the commutation relation
\[[\phi(t,\vec x),\pi(t,\vec y)]=i\delta^3(\vec x-\vec y)\]
\[[\phi(t,\vec x),\phi(t,\vec y)]=0=[\pi(t,\vec x),\pi(t,\vec y)]\]
Equivalently, these relation can be imposed saying that $\alpha$, $\alpha^\dagger$ are annihilation and creation operators that satisfy
\[[\alpha(\vec k),\alpha(\vec k')]=0=[\alpha(\vec k),\alpha(\vec k')]\]
\[[\alpha(\vec k), \alpha^\dagger(\vec k')]=\delta^3(\vec k-\vec k')\]

\skipline


Let's introduce an alternative notation in oder to obtain a covariant algebra for ladder operators and a covariant normalization for states.

We introduce the Lorentz invariant\footnote{We don't consider full Lorentz group but only transformations that preserve time, i.e. transformations with determinant equal to 1 (proper Lorentz transformations).} measure (under $SO^+(3,1)$)
\[\de^4k\,\delta(k^2+m^2)\theta(k^0)\]
{\small Notice that the theta function picks an arrow of time (i.e. particles or antiparticles).}

Using proprieties of delta function, we know that $\delta(k^2+m^2)$ can be splitted as follows (here $k^0\ne\omega$ in general)
\[\delta (k^2+m^2)=\frac1{2k^0}\left[\delta\left(k^0-\sqrt{m^2+\vec k^2}\right)+\delta\left(k^0+\sqrt{m^2+\vec k^2}\right)\right]\]

We then have (recall $\omega=\sqrt{m^2+\vec k^2}$)
\[\int\de k^0\delta(k^2+m^2)\theta(k^0)=\frac1{2\omega}\]
thus we see that if we take $f(k)\propto\omega$ then $\de^3k/f(k)$ will be Lorentz invariant. If we take $f(k)=(2\pi)^32\omega$ then the field express with the corresponding Lorentz invariant differential is
\begin{align*}
\phi(x)
&=\frac1{(2\pi^3)}\int\frac{\de^3 k}{2k^0}\left(a(k)e^{ikx}+a^\dagger(k)e^{-ikx}\right)
\end{align*}
As a consequence of commutation relations, ladder operators obey the following covariant algebra (obviously $k^0=\omega$)
\[[a(k),a(k')]=[a^\dagger(k),a^\dagger(k')]=0\]
\[[a(k),a^\dagger(k')]=(2\pi)^32k^0\delta^3(\vec k-\vec k')\]
We obtain a Fock space made of covariant normalized states
\[\braket0=1,\qquad\ket k=a^\dagger(k)\ket0\]
\[\braket{k}{k'}=(2\pi)^32k^0\delta^3(\vec k-\vec k')\]


In the free theory we can also write down $a$ and $a^\dagger$ in function of the field, in particular
\begin{equation}\label{eqn:ladder-field-relation}
a^\dagger(k)=-i\int\de^3x(e^{ikx}\partial_0\phi(x)-\partial_0e^{ikx}\phi(x))
\end{equation}

\begin{exercise}
Prove explicitly equation \eqref{eqn:ladder-field-relation}
\end{exercise}
\todo{insert solution}


\subsubsection{Scattering amplitude in interacting theory}
\textsf{Srednicki, chap 5}.\\

When we do a scattering experiment (which are function of momenta) $a$ and $a^\dagger$ will became time dependent operators. Consider a generic lagrangian in the form
\[\mathcal L=\mathcal L_0+\mathcal L_{int}\]
where $\mathcal L_0$ is the free field lagrangian while $\mathcal L_{int}$ is an interaction term.

Inital and final states takes the from
\[\ket i=\lim_{t\rightarrow -\infty}a^\dagger(k_1,t)\dots a^\dagger(k_n,t)\ket0\]
\[\ket f=\lim_{t\rightarrow +\infty}a^\dagger(k_1',t)\dots a^\dagger(k'_n,t)\ket0\]
This states must be normalized, for istance
\[\braket i=1=\braket f\]

Notice that now $a$ and $a^\dagger$ are ladder operators for the interacting theory, no more for the free theory. Now we look for an explicit expression for the scattering amplitude
\[S_{fi}=\braket {f}{i}\]
in the interaction formalism.
In order to obtain an useful expression for that we better try to express latter operators in terms of fields of the theory.
Let's start from the free field, we can write
\begin{equation}\label{eqn:ladder-inf-diff}\begin{split}
a^\dagger(k,+\infty)-a^\dagger(k, -\infty)
&=\int_{-\infty}^{+\infty}\de t\,\partial_0a^\dagger(k,t)\\
&=\int\de t\,\partial_0\left(-i\int_{-\infty}^{+\infty}\de^3x(e^{ikx}\partial_0\phi-\partial_0e^{ikx}\phi)\right)\\
&=-i\int\de^4x\,e^{ikx}(\partial_0^2+(k^0)^2)\phi(x)\\
&=-i\int\de^4x\,e^{ikx}(\partial_0^2+m^2+\vec k^2)\phi(x)\\
&=-i\int\de^4x\,[e^{ikx}(\partial_0^2+m^2)\phi(x)-\phi(x)\vect{\nabla}^2e^{ikx}]\\
&=-i\int\de^4x\,[e^{ikx}(\partial_0^2-\vect{\nabla}^2+m^2)\phi(x)] + \text{surface term}\\
&=-i\int\de^4x\,e^{ikx}(-\partial_\mu\partial^\mu+m^2)\phi(x)
\end{split}\end{equation}
and vanishes in the free theory because of the e.o.m. as we expected since we are considering a free field.

Let assume that \eqref{eqn:ladder-field-relation} holds also in the interaction case, then also the latter formula holds, but in this case the last term in general is not vanishing. 
Consider scattering amplitude
\[S_{fi}=\braket fi=\bra0a(k_1',+\infty)\dots a(k'_n,+\infty)a^\dagger(k_1,-\infty)\dots a^\dagger(k_n,-\infty)\ket0
\]
since operators are time ordered I can write
\[S_{fi}=\bra0T\left[a(k_1',+\infty)\dots a(k'_n,+\infty)a^\dagger(k_1,-\infty)\dots a^\dagger(k_n,-\infty)\right]\ket0
\]

Using \eqref{eqn:ladder-inf-diff} and its complex conjugated, i can rewrite:
\begin{mdframed}[style=mybox]
\begin{equation}\label{eqn:LSZ}\begin{split}
S_{fi}
&=i^{m+n}\int
\prod_{i=1}^m\de^4x_i\,e^{ik_ix_i}(-\square_{x_i}+m^2)
\prod_{j=1}^n\de^4y_j\,e^{ig_jk_j}(-\square_{y_j}+m^2)\times\\
&\hspace{0.5cm}\times
\underbrace{\bra0 T\left[\phi(x_1)\dots\phi(x_m)\phi(y_1)\dots\phi(y_n)\right]\ket0}_{G(x_1,\dots,x_n,y_1,\dots,y_n)}
\end{split}\end{equation}
\end{mdframed}

Eq. \eqref{eqn:LSZ} is the \textbf{Lehmann-Symanzik-Zimmermann reduction formula}, it express the scattering amplitude in terms of Green functions. Here is clear the importance of the Green functions.

Let's recall a crucial assumption we made: formula \eqref {eqn:ladder-field-relation} must holds in the interacting case.
The equation  \eqref {eqn:ladder-field-relation} holds for the interacting theory only if a couple of conditions are satisfied:
\begin{enumerate}
\item $\bra0\phi(x)\ket0=0$

This means that whenever you start from the state $\ket0$ the operator $a^\dagger$ must create some particle state, orthogonal to the vacuum. This is obvious for free theories, but does not holds in general for interacting theories. In some cases $\phi(x)\ket0$ is a linear combination of one particle states and the vacuum state. 

So, if $v=\bra0\phi(x)\ket0$ is a Lorentz invariant number different to zero, we will shift the field $\phi(x)$ by the constant $v$:
\[\phi(x)\rightarrow\phi(x)+v=\tilde \phi(x)\]
This is just a change in the name of the operator of interest, and does not affect the physics. However, the shifted $\tilde\phi(x)$ obeys, by costruction, $\bra0\tilde\phi(x)\ket0=0$.
\item $\bra k\phi(x)\ket 0=e^{-ikx}$

This is what it is in the free theory, and we know that in free theory, $a^\dagger(\pm\infty)$ creates a correctly normalized one-particle state. Thus, for $a^\dagger(\pm\infty)$ creates a correctly normalized one-particle state in the interacting theory, we must have $\bra k\phi(x)\ket 0=\bra ke^{-i\hat kx}\phi(0)e^{i\hat kx}\ket0=e^{-ikx}\bra k\phi(0)\ket 0=e^{-ikx}$.

So, if $\bra k\phi(0)\ket 0$ is a Lorentz invariant number different to one, we will rescale (or, \emph{renormalize} $\phi(x)$ by a multiplicative constant:
\[\phi(x)\rightarrow Z_{\phi}^{1/2}\phi=\phi_{int}\]
 This is just a change of the name of the operator of interest, and does not affect the physics. However, the rescaled $\phi(x)$ obeys, by costruction, $\bra k\phi(0)\ket 0=1$.
\end{enumerate}

Let us recap. The basic formula for scattering amplitude in terms of the fields of an interacting quantum field theory is the LSZ formula \eqref{eqn:LSZ}. The LSZ formula is valid \emph{provided} that the fields obeys
\begin{equation}\label{eqn:LSZ-condition}
\bra0\phi(x)\ket0=0\quad\text{and}\quad\bra k\phi(x)\ket0=e^{-ikx}
\end{equation}
These normalization conditions may conflict with our original choice of field and parameter normalization in the lagrangian. Consider, for example, a lagrangian originally specified as
\[\mathcal L=\underbrace{-\frac12\partial^\mu\phi\partial_\mu\phi-\frac12m^2\phi^2}_{\mathcal L_0}+\underbrace{\frac16g\phi^3}_{\mathcal L_{int}}\]
After shifting and rescaling (and renaming some parameters), we will have instead
\[\mathcal L'=\mathcal L+\mathcal L_{CT}=-\frac12Z_\phi\partial^\mu\phi\partial_\mu\phi-\frac12Z_mm^2\phi^2+\frac16Z_gg\phi^3+Y\phi\]
where
\[\mathcal L_{CT}=-\frac12(Z_\phi-1)\partial^\mu\phi\partial_\mu\phi-\frac12(Z_m-1)m^2\phi^2+\frac16(Z_g-1)g\phi^3+Y\phi\]
\todo{Verificare per cosa sta CT}
is a new term called \emph{countive (?) lagrangian}.
 Here $Z_\phi$, $Z_m$, $Z_g$ and $Y$ are yet unknown constant. They must be chosen to ensure the validity of eq.\eqref{eqn:LSZ-condition}; this gives us two conditions in four unknown. We fix the parameter $Z_mm^2$ by requiring  to be equal to the actual mass of the particle (equivalently, the energy of the first excited state relative to the ground state), and we fix the parameter $Z_gg$ by requiring some particular scattering cross section to depend on the coupling term in some particular way. So we have four conditions in four unknown, and it is possible to calculate $Y$ and the three $Z$s order by order in powers of $g$.

Next, we must develop the tools needed to compute the correlation functions $\bra0 T\left[\phi(x_1)\dots\right]\ket0$ in an interacting quantum field theory.















\end{document}