\usepackage[T1]{fontenc}
\usepackage[utf8]{inputenc}
\usepackage[english]{babel}
\usepackage{geometry}
	\geometry{a4paper, top=3cm, bottom=3cm, left=2.5cm, right=2.5cm}
\usepackage{amsfonts}
\usepackage{amsmath}
\usepackage{latexsym}
\usepackage{cases}
\usepackage{mathtools}
\usepackage{slashed}
\usepackage{braket}
\usepackage{mathrsfs}
\usepackage{quoting}
\usepackage{bbold} %identity symbol (double 1)
\usepackage[framemethod=TikZ, nobreak=true]{mdframed}
\usepackage{amsthm}
\usepackage[compat=1.1.0]{tikz-feynman}
\usepackage{enumitem}
	\setlist{nosep}
	\setlist{label=(\roman*)}
\usepackage[framemethod=TikZ]{mdframed}
	\usepackage{amsthm}
\usepackage{booktabs}
\usepackage{caption}
\usepackage{dsfont}
\usepackage{float}

%footnote con num romani invece dei numeri (lo trovo più chiaro visto che potrebbero essere scambiati per apici, però vedi tu se tenerlo o no)
\renewcommand{\thefootnote}
	{\Roman{footnote}}

\renewcommand{\epsilon}{\varepsilon}
\renewcommand{\phi}{\varphi}
\renewcommand{\vec}{\mathbf}

\newcommand{\mathL}{\mathcal{L}}
\newcommand{\mathH}{\mathcal{H}}
\newcommand{\mathM}{\mathcal{M}}
\newcommand{\mathR}{\mathbb{R}} %Reali
\newcommand{\mathZ}{\mathbb{Z}} %Interi
\newcommand{\de}{\textup{d}}
\newcommand{\so}{\Rightarrow}
\newcommand{\abs}[1]{\left\lvert #1 \right\rvert}
\newcommand{\norma}[1]{\left\lVert #1 \right\rVert}
\newcommand{\id}{\mathbb 1} %Identità con il doppio 1
\newcommand{\varid}{\textup {id}} %Identità con "id"
\newcommand{\Tr}{\textup{Tr}} %trace


\newcommand{\skipline}{\vspace{\baselineskip}} %Salta una riga (devi saltare una riga prima di mettere il comando)
\newcommand*\circled[1]{\tikz[baseline=(char.base)]{
            \node[shape=circle,draw,inner sep=2pt] (char) {#1};}}%numeri cerchiati
            
\usepackage[colorinlistoftodos,textsize=tiny]{todonotes}% TO DO points

%Struttura modulare
\usepackage{subfiles}
\newcommand{\onlyinsubfile}[1]{#1}
\newcommand{\onlyinmainfile}[1]{}

%Example
\mdfdefinestyle{example}{%
linecolor=gray,linewidth=2pt,%
frametitlerule=true,%
frametitlebackgroundcolor=gray!20,
innertopmargin=\topskip,
}
\mdtheorem[style=example, nobreak=false]{example}{Example}
%%%%%%%%%%%%%%%%%%%%%%%%%%%%%%

%Exercise
% metti il colore che preferisci su linecolor e frametitlebackgroundcolor
% nel file .tex: \begin{exercise}[TITOLO ESERICZIO], se non vuoi dare un titolo all'esercizio non mettere le parentesi quadre (lo stesso vale per example)
\definecolor{blue}{RGB}{0, 114, 178} % modifica qui se vuoi creare un colore personalizzato con codice RGB

\mdfdefinestyle{exercise}{%
linecolor=blue,linewidth=2pt,%	
frametitlerule=true,%
frametitlebackgroundcolor=blue!20,	% !20 = saturazione del colore
innertopmargin=\topskip,
}
\mdtheorem[style=exercise, nobreak=false]{exercise}{Exercise}
%%%%%%%%%%%%%%%%%%%%%%%%%%%%%%

%Framed
\mdfdefinestyle{mybox}{
	leftmargin = 1cm,
	rightmargin = 1cm,
	innerleftmargin = 5 pt,
	innerrightmargin = 5 pt,
	font = \itshape\rmfamily,
}