\usepackage[T1]{fontenc}
\usepackage[utf8]{inputenc}
\usepackage[english]{babel}
\usepackage{geometry}
	\geometry{a4paper, top=3cm, bottom=3cm, left=2.5cm, right=2.5cm}
\usepackage{amsfonts}
\usepackage{amsmath}
\usepackage{latexsym}
\usepackage{cases}
\usepackage{mathtools}
\usepackage{slashed}
\usepackage{physics}
\usepackage{mathrsfs}
\usepackage{quoting}
\usepackage{bbold} %identity symbol (double 1)
\usepackage[framemethod=TikZ, nobreak=true]{mdframed}
\usepackage{amsthm}
\usepackage[compat=1.1.0]{tikz-feynman}
\usepackage{enumitem}
	\setlist{nosep}
	\setlist{label=(\roman*)}
\usepackage[framemethod=TikZ]{mdframed}%ambient thm ecc...
	\usepackage{amsthm}
\usepackage{booktabs}
\usepackage{caption}
\usepackage{dsfont}
\usepackage{float}
\usepackage{tikz}
\usepackage{mathdots}
\usepackage{yhmath}
\usepackage{cancel}
\usepackage{color}
\usepackage{siunitx}
\usepackage{array}
\usepackage{multirow}
\usepackage{amssymb}
\usepackage{gensymb}
\usepackage{tabularx}
\usepackage{booktabs}
\usetikzlibrary{fadings}
\usetikzlibrary{patterns}
\usetikzlibrary{shadows.blur}
\usepackage{hyperref}
\hypersetup{
	linktoc=all,
	colorlinks=true,
	linkcolor={blue!70!black},
	citecolor={green}
}


%footnote con num romani invece dei numeri (lo trovo più chiaro visto che potrebbero essere scambiati per apici)
\renewcommand{\thefootnote}
	{\Roman{footnote}}

%scambia phi e varphi
%\let\temp\phi
%\let\phi\varphi
%\let\varphi\temp

%scambia epsilon e varepsilon
\let\temp\epsilon
\let\epsilon\varepsilon
\let\varepsilon\temp

\let\vect\vec
\let\vec\mathbf

\newcommand{\p}[1]{\left(#1\right)}
\newcommand{\RR}{\mathbb{R}} %Reali
\newcommand{\CC}{\mathbb{C}} %Complessi
\newcommand{\ZZ}{\mathbb{Z}} %Interi
\newcommand{\de}{\textup{d}}
\newcommand{\so}{\rightarrow}
\newcommand{\id}{\mathbb 1} %Identità con il doppio 1
\newcommand{\varid}{\textup {id}} %Identità con "id"
\newcommand{\der}[2]{\frac{\de #1}{\de #2}}
\newcommand{\dder}[3]{\frac{\de^2 #1}{\de #2\,\de #3}}
\newcommand{\pder}[2]{\frac{\partial #1}{\partial #2}}
\newcommand{\pdder}[3]{\frac{\partial^2 #1}{\partial #2\,\partial #3}}
\newcommand{\inv}[1]{{#1}^{-1}}

\newcommand{\comma}{\quad,\quad}

\newenvironment{eqalign}{\begin{equation}\begin{aligned}}{\end{aligned}\end{equation}}

\newcommand{\skipline}{\leavevmode\vspace{\baselineskip}} %Salta una riga (devi saltare una riga prima di mettere il comando)

\newcommand*\circled[1]{\tikz[baseline=(char.base)]{
            \node[shape=circle,draw,inner sep=2pt] (char) {#1};}}%numeri cerchiati
            
\usepackage[colorinlistoftodos,textsize=tiny]{todonotes}% TO DO points

%Struttura modulare
\usepackage{subfiles}
\newcommand{\onlyinsubfile}[1]{#1}
\newcommand{\onlyinmainfile}[1]{}

%Example
\mdfdefinestyle{example}{%
linecolor=gray,linewidth=2pt,%
frametitlerule=true,%
frametitlebackgroundcolor=gray!20,
innertopmargin=\topskip,
}
\mdtheorem[style=example, nobreak=false]{example}{Example}
%%%%%%%%%%%%%%%%%%%%%%%%%%%%%%

%Exercise
% metti il colore che preferisci su linecolor e frametitlebackgroundcolor
% nel file .tex: \begin{exercise}[TITOLO ESERICZIO], se non vuoi dare un titolo all'esercizio non mettere le parentesi quadre (lo stesso vale per example)
\definecolor{blue}{RGB}{0, 114, 178} % modifica qui se vuoi creare un colore personalizzato con codice RGB

\mdfdefinestyle{exercise}{%
linecolor=blue,linewidth=2pt,%	
frametitlerule=true,%
frametitlebackgroundcolor=blue!20,	% !20 = saturazione del colore
innertopmargin=\topskip,
}
\mdtheorem[style=exercise, nobreak=false]{exercise}{Exercise}
%%%%%%%%%%%%%%%%%%%%%%%%%%%%%%

%Framed
\mdfdefinestyle{mybox}{
	leftmargin = 1cm,
	rightmargin = 1cm,
	innerleftmargin = 5 pt,
	innerrightmargin = 5 pt,
	font = \itshape\rmfamily,
}
% use \begin{mdframed}[style=mybox] -> \end{mdframed}

\usepackage[backend=biber, style=numeric, sorting=nty]{biblatex}
\addbibresource{Bibliography.bib}